% Options for packages loaded elsewhere
\PassOptionsToPackage{unicode}{hyperref}
\PassOptionsToPackage{hyphens}{url}
%
\documentclass[
  man,floatsintext]{apa6}
\usepackage{amsmath,amssymb}
\usepackage{lmodern}
\usepackage{iftex}
\ifPDFTeX
  \usepackage[T1]{fontenc}
  \usepackage[utf8]{inputenc}
  \usepackage{textcomp} % provide euro and other symbols
\else % if luatex or xetex
  \usepackage{unicode-math}
  \defaultfontfeatures{Scale=MatchLowercase}
  \defaultfontfeatures[\rmfamily]{Ligatures=TeX,Scale=1}
\fi
% Use upquote if available, for straight quotes in verbatim environments
\IfFileExists{upquote.sty}{\usepackage{upquote}}{}
\IfFileExists{microtype.sty}{% use microtype if available
  \usepackage[]{microtype}
  \UseMicrotypeSet[protrusion]{basicmath} % disable protrusion for tt fonts
}{}
\makeatletter
\@ifundefined{KOMAClassName}{% if non-KOMA class
  \IfFileExists{parskip.sty}{%
    \usepackage{parskip}
  }{% else
    \setlength{\parindent}{0pt}
    \setlength{\parskip}{6pt plus 2pt minus 1pt}}
}{% if KOMA class
  \KOMAoptions{parskip=half}}
\makeatother
\usepackage{xcolor}
\usepackage{longtable,booktabs,array}
\usepackage{calc} % for calculating minipage widths
% Correct order of tables after \paragraph or \subparagraph
\usepackage{etoolbox}
\makeatletter
\patchcmd\longtable{\par}{\if@noskipsec\mbox{}\fi\par}{}{}
\makeatother
% Allow footnotes in longtable head/foot
\IfFileExists{footnotehyper.sty}{\usepackage{footnotehyper}}{\usepackage{footnote}}
\makesavenoteenv{longtable}
\usepackage{graphicx}
\makeatletter
\def\maxwidth{\ifdim\Gin@nat@width>\linewidth\linewidth\else\Gin@nat@width\fi}
\def\maxheight{\ifdim\Gin@nat@height>\textheight\textheight\else\Gin@nat@height\fi}
\makeatother
% Scale images if necessary, so that they will not overflow the page
% margins by default, and it is still possible to overwrite the defaults
% using explicit options in \includegraphics[width, height, ...]{}
\setkeys{Gin}{width=\maxwidth,height=\maxheight,keepaspectratio}
% Set default figure placement to htbp
\makeatletter
\def\fps@figure{htbp}
\makeatother
\setlength{\emergencystretch}{3em} % prevent overfull lines
\providecommand{\tightlist}{%
  \setlength{\itemsep}{0pt}\setlength{\parskip}{0pt}}
\setcounter{secnumdepth}{-\maxdimen} % remove section numbering
% Make \paragraph and \subparagraph free-standing
\ifx\paragraph\undefined\else
  \let\oldparagraph\paragraph
  \renewcommand{\paragraph}[1]{\oldparagraph{#1}\mbox{}}
\fi
\ifx\subparagraph\undefined\else
  \let\oldsubparagraph\subparagraph
  \renewcommand{\subparagraph}[1]{\oldsubparagraph{#1}\mbox{}}
\fi
\newlength{\cslhangindent}
\setlength{\cslhangindent}{1.5em}
\newlength{\csllabelwidth}
\setlength{\csllabelwidth}{3em}
\newlength{\cslentryspacingunit} % times entry-spacing
\setlength{\cslentryspacingunit}{\parskip}
\newenvironment{CSLReferences}[2] % #1 hanging-ident, #2 entry spacing
 {% don't indent paragraphs
  \setlength{\parindent}{0pt}
  % turn on hanging indent if param 1 is 1
  \ifodd #1
  \let\oldpar\par
  \def\par{\hangindent=\cslhangindent\oldpar}
  \fi
  % set entry spacing
  \setlength{\parskip}{#2\cslentryspacingunit}
 }%
 {}
\usepackage{calc}
\newcommand{\CSLBlock}[1]{#1\hfill\break}
\newcommand{\CSLLeftMargin}[1]{\parbox[t]{\csllabelwidth}{#1}}
\newcommand{\CSLRightInline}[1]{\parbox[t]{\linewidth - \csllabelwidth}{#1}\break}
\newcommand{\CSLIndent}[1]{\hspace{\cslhangindent}#1}
\ifLuaTeX
\usepackage[bidi=basic]{babel}
\else
\usepackage[bidi=default]{babel}
\fi
\babelprovide[main,import]{english}
% get rid of language-specific shorthands (see #6817):
\let\LanguageShortHands\languageshorthands
\def\languageshorthands#1{}
% Manuscript styling
\usepackage{upgreek}
\captionsetup{font=singlespacing,justification=justified}

% Table formatting
\usepackage{longtable}
\usepackage{lscape}
% \usepackage[counterclockwise]{rotating}   % Landscape page setup for large tables
\usepackage{multirow}		% Table styling
\usepackage{tabularx}		% Control Column width
\usepackage[flushleft]{threeparttable}	% Allows for three part tables with a specified notes section
\usepackage{threeparttablex}            % Lets threeparttable work with longtable

% Create new environments so endfloat can handle them
% \newenvironment{ltable}
%   {\begin{landscape}\centering\begin{threeparttable}}
%   {\end{threeparttable}\end{landscape}}
\newenvironment{lltable}{\begin{landscape}\centering\begin{ThreePartTable}}{\end{ThreePartTable}\end{landscape}}

% Enables adjusting longtable caption width to table width
% Solution found at http://golatex.de/longtable-mit-caption-so-breit-wie-die-tabelle-t15767.html
\makeatletter
\newcommand\LastLTentrywidth{1em}
\newlength\longtablewidth
\setlength{\longtablewidth}{1in}
\newcommand{\getlongtablewidth}{\begingroup \ifcsname LT@\roman{LT@tables}\endcsname \global\longtablewidth=0pt \renewcommand{\LT@entry}[2]{\global\advance\longtablewidth by ##2\relax\gdef\LastLTentrywidth{##2}}\@nameuse{LT@\roman{LT@tables}} \fi \endgroup}

% \setlength{\parindent}{0.5in}
% \setlength{\parskip}{0pt plus 0pt minus 0pt}

% Overwrite redefinition of paragraph and subparagraph by the default LaTeX template
% See https://github.com/crsh/papaja/issues/292
\makeatletter
\renewcommand{\paragraph}{\@startsection{paragraph}{4}{\parindent}%
  {0\baselineskip \@plus 0.2ex \@minus 0.2ex}%
  {-1em}%
  {\normalfont\normalsize\bfseries\itshape\typesectitle}}

\renewcommand{\subparagraph}[1]{\@startsection{subparagraph}{5}{1em}%
  {0\baselineskip \@plus 0.2ex \@minus 0.2ex}%
  {-\z@\relax}%
  {\normalfont\normalsize\itshape\hspace{\parindent}{#1}\textit{\addperi}}{\relax}}
\makeatother

% \usepackage{etoolbox}
\makeatletter
\patchcmd{\HyOrg@maketitle}
  {\section{\normalfont\normalsize\abstractname}}
  {\section*{\normalfont\normalsize\abstractname}}
  {}{\typeout{Failed to patch abstract.}}
\patchcmd{\HyOrg@maketitle}
  {\section{\protect\normalfont{\@title}}}
  {\section*{\protect\normalfont{\@title}}}
  {}{\typeout{Failed to patch title.}}
\makeatother

\usepackage{xpatch}
\makeatletter
\xapptocmd\appendix
  {\xapptocmd\section
    {\addcontentsline{toc}{section}{\appendixname\ifoneappendix\else~\theappendix\fi\\: #1}}
    {}{\InnerPatchFailed}%
  }
{}{\PatchFailed}
\keywords{keywords\newline\indent Word count: X}
\usepackage{lineno}

\linenumbers
\usepackage{csquotes}
\ifLuaTeX
  \usepackage{selnolig}  % disable illegal ligatures
\fi
\IfFileExists{bookmark.sty}{\usepackage{bookmark}}{\usepackage{hyperref}}
\IfFileExists{xurl.sty}{\usepackage{xurl}}{} % add URL line breaks if available
\urlstyle{same} % disable monospaced font for URLs
\hypersetup{
  pdftitle={Symptom-related information changes the decision-making policy in eating disorders},
  pdfauthor={Corrado Caudek1 \& Ernst-August Doelle1,2},
  pdflang={en-EN},
  pdfkeywords={keywords},
  hidelinks,
  pdfcreator={LaTeX via pandoc}}

\title{Symptom-related information changes the decision-making policy in eating disorders}
\author{Corrado Caudek\textsuperscript{1} \& Ernst-August Doelle\textsuperscript{1,2}}
\date{}


\shorttitle{Title}

\authornote{

Add complete departmental affiliations for each author here. Each new line herein must be indented, like this line.

Enter author note here.

The authors made the following contributions. Corrado Caudek: Conceptualization, Writing - Original Draft Preparation, Writing - Review \& Editing; Ernst-August Doelle: Writing - Review \& Editing, Supervision.

Correspondence concerning this article should be addressed to Corrado Caudek, Postal address. E-mail: \href{mailto:my@email.com}{\nolinkurl{my@email.com}}

}

\affiliation{\vspace{0.5cm}\textsuperscript{1} Wilhelm-Wundt-University\\\textsuperscript{2} Konstanz Business School}

\abstract{%
One or two sentences providing a \textbf{basic introduction} to the field, comprehensible to a scientist in any discipline.

Two to three sentences of \textbf{more detailed background}, comprehensible to scientists in related disciplines.

One sentence clearly stating the \textbf{general problem} being addressed by this particular study.

One sentence summarizing the main result (with the words ``\textbf{here we show}'' or their equivalent).

Two or three sentences explaining what the \textbf{main result} reveals in direct comparison to what was thought to be the case previously, or how the main result adds to previous knowledge.

One or two sentences to put the results into a more \textbf{general context}.

Two or three sentences to provide a \textbf{broader perspective}, readily comprehensible to a scientist in any discipline.
}



\begin{document}
\maketitle

\hypertarget{introduction}{%
\section{Introduction}\label{introduction}}

Eating disorders (EDs) are prevalent in adolescents and young adults, affecting up to 15\% of women and 5\% of men. They cause physical harm and impair psychosocial functioning, with a five- to six-fold increased risk of suicide attempts compared to those without EDs (Udo, Bitley, \& Grilo, 2019) and high mortality rates in cases of anorexia nervosa (AN) -- (Qian et al., 2022). Effective treatment of EDs is challenging (Chang, Delgadillo, \& Waller, 2021), making it important to gain a better understanding of the underlying mechanisms.

Executive dysfunction has been frequently suggested as a potential risk factor and maintaining factor for the disease (cognitive inflexibility impairments: Wu et al., 2014; decision-making impairments: Guillaume et al., 2015; inhibitory-control impairments: Bartholdy, Dalton, O'Daly, Campbell, \& Schmidt, 2016). Studies of aberrant executive processes in EDs have often focused on cognitive inflexibility, using a reinforcement learning (RL) paradigm. The theory of maladaptive associative learning being a cause of EDs is intriguing, as it suggests possible treatment options, but the evidence to support it is inconsistent (for a recent discussion, see Caudek, Sica, Cerea, Colpizzi, \& Stendardi, 2021). This study aims to explore if ED patients can have flawed decision-making despite having normal cognitive decision-making skills. It tests the hypothesis that \emph{task-irrelevant} symptom information can negatively impact decision-making in EDs, potentially indicating that disordered eating may not stem from deficient decision-making abilities, but rather from external factors like long-term goals, personality traits, etc. affecting their choices. The potential translational impact of this result would be noteworthy, when considering that \ldots{}

\hypertarget{influence-of-outcome-irrilevant-variables-on-rl}{%
\subsection{Influence of outcome-irrilevant variables on RL}\label{influence-of-outcome-irrilevant-variables-on-rl}}

RL is the ability to infer causal associations between actions and outcomes in a trial-and-error manner. Learning the consequences of past actions is usually studied in the laboratory with a 2-armed bandit task, where a decision maker is presented with two options. One option has a higher likelihood of winning. The participant must learn which choice will yield the highest reward.

In the 2-armed bandit task, the best strategy for maximizing long-term rewards is based solely on the history of actions and outcomes. Recent research has shown, however, that human reinforcement learning can be impacted by features unrelated to the outcomes. For instance, a study by Shahar et al. (2019) explored the impact of spatial-motor associations on participant reinforcement learning. Optimal decision making should prioritize the reward regardless of any spatial-motor associations (such as the choice of response key in the previous trial).
Instead, Shahar et al. (2019) found that rewards had a greater impact on the probability of selecting one of two images presented in each trial when the chosen image was linked to the same response key in both the n-1 and n trials. This demonstrates that, in the general population, decision making can be influenced by features that have no relation to the outcomes {[}\emph{i.e.}, the image/effector response mapping when only the image identity was predictive of the reward; see also Ben-Artzi, Luria, and Shahar (2022){]}.

The demonstration of outcome-irrelevant features affecting action value-updating raises the possibility of reevaluating previous reinforcement learning findings in EDs. The subpar decision-making in EDs might be attributed to the influence of these factors, instead of solely being viewed as a deficit. This extraneous influence could, at least in part, explain why aberrant decision making has been observed in some EDs studies, but not in others (for a discussion, see Caudek et al., 2021).

We posit that, when they are asked to choose between a food or a non-food item in a 2-bandit task, AN patients (given their rigid weight-control behavior and the importance attributed to the long-term goal of thinness) and BN patients (given their impulsivity) are affected by the interference deriving from the processing of food-related information. Therefore, we hypothesize that long-term goals (in AN) or temperamental factors (in BN) can lead to an altered decision-making process in EDs, when the food/not food dimension is present in the task but is outcome-irrelevant, \emph{even in the absence of any decision-making deficit} (see also Haynos, Widge, Anderson, \& Redish, 2022).

We make two specific predictions concerning the effects of outcome-irrelevant features on PRL performance (domain-specific cognitive load hypothesis). First, we expect food information to be processed in a more conservative manner than neutral information, for both ED patients and HCs. This result, never before observed in a PRL task, would be consistent with the differences in attention orienting and cognitive control mechanisms for food and non-food information that had been previously reported in other tasks (\emph{e.g.}, Schiff, Testa, Rusconi, Angeli, \& Mapelli, 2021). Second, and more importantly, we predict a decrease in learning-rate with the symptom-induced interference evoked by disease-specific, but \emph{outcome-irrelevant}, information (domain-specific policy hypothesis).

\hypertarget{methods}{%
\section{Methods}\label{methods}}

We report how we determined our sample size, all data exclusions (if any), all manipulations, and all measures in the study.

\hypertarget{participants}{%
\subsection{Participants}\label{participants}}

The final sample consists of 69 female outpatients (acAN N = 40, recAN N = 10, acBN N = 13, recBN = 6) and 222 healthy female controls (HCs). Outpatients met Diagnostic and Statistical Manual of Mental Disorders-5 (DSM-5) (American Psychiatric Association, 2013) criteria for AN or BN. They were recruited from the Specchidacqua Institute, Montecatini (PT), Italy, specialized in Eating Disorders. Eligibility was evaluated by the Mental Health professionals of the Institute, the exclusion criteria were having neurological illness, suicidal ideation, alcohol or drug addiction, or psychosis. The acAN (mean age = 20.5 years, \emph{SD} = 1.13) and acBN (mean age = 23.15 years, \emph{SD} = 1.87) participants were admitted to psychological treatment at Specchidacqua Institute, 45\% of them were also taking antidepressant medication (SSRI), and 38\% reported comorbidity with other psychiatric illnesses (22\% anxiety disorders, 20\% obsessive-compulsive disorder, 9\% mood disorders). Mean Body Mass Index was considerable lower for acAN patients (BMI mean = 18.29kg/m2 ) then acBN (BMI mean = 24.84kg/m2). Recovered outpatients were recruited from the Gruber Residence, Bologna (BO), Italy. To be included in the recovered group, recAN (mean age = 24.1 years, \emph{SD} = 1.8) and recBN (mean age = 29.3 years, \emph{SD} = 2.5) outpatients had to (a) not being seriously underweight (\(BMI \ge 18.5\) kg/m2), (b) not engage in dysfunctional eating behaviors (\emph{e.g.}, restrictive diet or binging/purging) for at least 6 months, and (c) being adherent to the psychological treatment. HC participants were recruited from undergraduate psychology courses at the University of Florence, Italy, or via social networks. To be included in the HC group, participants had to have a normal Body Mass Index (BMI mean = 21.29 kg/m2), have no history of psychiatric illness, and have no diagnosis of Eating Disorders, according to the Eating Attitudes Test-26 {[}EAT-26; Garner, Olmsted, Bohr, and Garfinkel (1982), Dotti and Lazzari (1998){]} score (EAT-26 \textless{} 20). However, 28 out of 222 participants exceeded the EAT-26 cut-off (EAT-26 \textgreater{} 20), meaning the presence of a tendency to eating symptoms. Therefore, the final HC group was composed by 194 participants (mean age = 21.5 years, \emph{SD} = 0.23), and the other 28 were classified as at-risk participants (mean age = 21.28 years, \emph{SD} = 0.55).
All participants were caucasian, right-handed, had a normal intelligence level (measured by the Progressive Raven's Matrices Intelligence test), and were na"ive to the aim of the study.

\hypertarget{material}{%
\subsection{Material}\label{material}}

\hypertarget{clinical-and-demographic-measurements}{%
\subsubsection{Clinical and Demographic Measurements}\label{clinical-and-demographic-measurements}}

The \emph{Eating Attitude Test-26} (EAT-26, Garner et al., 1982) consists of 26 items assessing levels and types of eating disturbances in the past three mouths. The EAT-26 is characterized by three subscales: the Dieting Scale, the Bulimia and Food Preoccupation Scale and the Oral Control Scale. Scores \(\ge 20\) point out the presence of an eating disorder. Respondents are required to rate intensity associated with the items on a 6-point Likert scale (0 = never, rarely, sometimes; 3 = always).
The Italian version of the EAT-26 demonstrated good psychometric properties (Dotti \& Lazzari, 1998). In fact, Cronbach's alpha was high in an undergraduate sample for the Dieting scale (.87), for Bulimia and Food Preoccupation scale (.70), for Oral Control Scale (.62).
Cronbach's alpha for the total scores was 0.86.

The \emph{Body Shape Questionnaire-14} {[}BSQ-14; Dowson and Henderson (2001){]} is a 14-item self-report scale assessing the global body satisfaction in the past two weeks. Respondents are required to rate intensity of concerns about own appearance associated with the items on a 6-point Likert scale (1 = never, 6 = always). The Italian version of the BSQ-14 demonstrated good psychometric properties (Matera, Nerini, \& Stefanile, 2013). Cronbach's alpha was high (.93).

The \emph{Social Interaction Anxiety Scale} {[}SIAS; Mattick and Clarke (1998){]} is a 19-item self-report questionnaire assessing social interaction anxiety. Respondents are required to rate intensity associated with the items on a 4-point Likert scale from 0 (not at
all true) to 4 (extremely true). Total scores range from 0 to 76 and higher scores denote greater social interaction anxiety levels. Both original version and the Italian version (Sica, Musoni, Bisi, Lolli, \& Sighinolfi, 2007) show acceptable psychometric properties.

The \emph{Depression Anxiety Stress Scale-21} {[}DASS-21; Lovibond and Lovibond (1995){]} is a 21-item self-report measure assessing depression, anxiety, and stress over the previous week. Items are rated on a 4-point scale ranging from 0 (did not apply to me at all) to 3 (applied to me very much). Both the original and the Italian version (Bottesi et al., 2015) demonstrate adequate reliability (\(\alpha\) = 0.90).

The \emph{Rosenberg Self-Esteem Scale} {[}RSES; Rosenberg (1965){]} is a 10-item scale designed to assess person's overall self-esteem. It comprises five straightforwardly worded and five reverse-worded items each rated on a 4-point Likert scale ranging from 4 (strongly agree) to 1 (strongly disagree). The RSES demonstrate high internal consistency (alpha = .88) Rosenberg (1965) and good test--retest reliability (r = .82) (Fleming \& Courtney, 1984).

The \emph{Multidimensional Perfectionism Scale} {[}MPS-F; Frost, Marten, Lahart, and Rosenblate (1990){]} is a 35-item assessing perfectionism tendencies. The original version of the MPS-F comprised six sub-scales: Dimensions of Concern over Mistakes, Personal Standards, Parental Expectations, Parental Criticism, Doubts about Action, and Organization. However, it has been argued that the six-scales division caused a factorial instability. Therefore, Stoeber et al.~(1998) proposed that a better approach was to consider MPS-F as composed of four underlying factors: Concerns over Mistakes and Doubts (CMD), Parental Expectations and Criticism (PEC), Personal Standards (PS), and Organization (O). Both the original MPS-F and the Italian version (Lombardo, 2008) demonstrate adequate reliability.

\hypertarget{procedure}{%
\subsection{Procedure}\label{procedure}}

The study was approved by the Ethical Committee of the University of Florence, and was run in accordance with the Declaration of Helsinki.
Each eligible participant signed the informed consent and agreed to be part of the study.
Both the HCs group and the patients group completed the same tasks.
Data collection started in December, 2020 until June, 2022. We have to deal with COVID-19 restrictions for the most of the time. Thus, we collected data from HCs remotely: we recruited HCs participants by means of social networks or advertisements at the University. Interested people contacted us using the email on the advertisement, then we send them the informed consent, which they had to sign and send it back to us. Individuals that signed the informant consent were tested for eligibility using self-reported measures. Participants who met the inclusion criteria for HCs group, received instructions via email and completed the PRL task remotely. After completing the task, participants had to notify us, so that we can check the correct registration of data.
On the contrary, data collection for the clinical group was in person. We enrolled only eligible patients, selected by the mental health professionals of the Institute. We scheduled two meeting per participants at the Specchidacqua Institute, Montecatini (PT), Italy.\\
On the first session, participants signed the informed consent form and completed a battery of self-report questionnaires. On the second session, participants were asked to complete the PRL task. Data collection required overall 1 hour of their time.

Participants completed a reinforcement learning bandit task in two conditions: neutral (two neutral images on each trial) and symptom-specific (a symptom-specific and a neutral image on each trial). This design allowed us to examine outcome-irrelevant learning associated to a symptom-specific context.

Participants completed a total of 2 blocks of the reinforcement learning task. Each block included a different set of image stimuli and had XX trials. Participants did not received any bonus at the end of the task based on their performance.

For measuring cognitive flexibility, participants completed a computerized Probabilistic Reversal Learning (PRL) task.
There were two blocks of trials including 160 trials each. In one of the two blocks a neutral image (\emph{e.g.}, a lamp) and a symptom-related image (\emph{i.e.}, a piece of cake) were shown together, to test the domain-specificity hypothesis Caudek, Sica, Marchetti, Colpizzi, \& Stendardi, 2020). The other block included neutral images only, as a control task.
In both blocks we asked participants to choose one of two stimuli presented simultaneously on the left and right side of the center of a screen and made their choice with a keypress. They had 3s response time per trial. An image of a euro coin was provided as a reward and a strikethrough image of a euro coin as a punishment. Feedback was presented for 2 s.
The PRL comprises four epochs (\emph{e.g.}, a sequence of trials in which the same image was considered correct) of 40 trials each.
The feedback was probabilistic, which means that for each epoch the correct image was rewarded in the 70\% of the cases, whereas on 30\% of the trials participants received a negative feedback.
As a consequence, the other image provided no-reward 70\% of the time. Both blocks consisted of three rule changes (reversal phase).
Participants' aim was to earn as much money as possible. They were informed that the stimulus-reward contingencies would change, but they were not told how or when it would happen.
Total reward earned was shown at the end of each block.
The experiment was controlled by \href{https://www.psytoolkit.org/}{Psytoolkit}.

\hypertarget{data-analysis}{%
\subsection{Data analysis}\label{data-analysis}}

Credible effects were revealed by 95\% credible intervals or by 97.5\% of posterior samples falling above or below 0 when computing proportion of posterior in direction of effect.

\hypertarget{results}{%
\section{Results}\label{results}}

\hypertarget{quality-control}{%
\subsection{Quality Control}\label{quality-control}}

Trials were excluded for extreme RTs (\textless150 ms, \textgreater2500 ms), or if
the remaining (log transformed) RT exceeded the participant's
mean ± 3S.D. Participants' datasets were excluded if, in any
block, there were more than 20 RT outliers, fewer than 24 rich
or 7 lean rewards, a rich-to-lean reward ratio lower than 2.5, or
lower than 40\% correct accuracy. In Study 1, 258 depressed adults
and 36 controls passed the QC criteria. Study 2 data are from participants who passed these QC checks.

\hypertarget{estimating-outcome-irrelevant-learning}{%
\subsection{Estimating outcome-irrelevant learning}\label{estimating-outcome-irrelevant-learning}}

\hypertarget{spatial-motor-associations}{%
\subsubsection{Spatial-motor associations}\label{spatial-motor-associations}}

We start by examining the presence of spatial-motor associations on participants' choices. We found strong evidence for spatial-motor outcome-irrelevant learning: The difference in `stay' probability between previously rewarded and previously unrewarded response was larger for `same' (.427) than for `flipped' (.218) response/key mapping (posterior \(\beta\) = 0.93, \emph{SE} = 0.06, \(\text{HDI}_{.95}\) = {[}0.81, 1.06{]}; probability of direction (pd) 1.0; 0\% in ROPE (-0.10, 0.10) and Bayes Factor (BF) of \(>\) 100 against the null; Fig. 1). These results replicate those found by Shahar et al. (2019) and Ben-Artzi et al. (2022). There was no group (HC, AN, BN, RI) \(\times\) previous outcome \(\times\) mapping interaction (see Supplementary Materials).

\hypertarget{reinforcement-learning-and-drift-diffusion-modeling}{%
\subsection{Reinforcement learning and drift diffusion modeling}\label{reinforcement-learning-and-drift-diffusion-modeling}}

To capture the drift towards a two-choice decision (image A and image B) over time, we employed a hierarchical reinforcement learning drift diffusion model (RLDDM; Pedersen et al., 2017; Pedersen and Frank, 2020). The RLDDM was estimated in a hierarchical Bayesian framework using the \(\texttt{HDDMrl}\) module of the \(\texttt{HDDM}\) (version 0.9.7) Python package (Fengler et al., 2021; Wiecki et al., 2013).

RLDDM has six basic parameters: positive learning rate (\(alpha^+\)), negative learning rate (\(alpha^-\)), drift rate (\(v\)), decision threshold (\(a\)), non-decision time (\(t\)), and starting point bias (\(z\)) parameters. The \(\alpha\) parameter quantifies the learning rate in the Rescorla-Wagner delta learning rule (Rescorla, 1972); a higher learning rate results in rapid adaptation to reward expectations, while a lower learning rate results in slow adaptation. The parameter \(\alpha^+\) is computed from reinforcements, whereas \(\alpha^+\) is computed from punishments. The drift rate \(v\) is the average speed of evidence accumulation toward one decision. The decision boundary is the distance between two decision thresholds; an increase of \(a\) increases the evidence needed to make a decision. The increase of \(a\) leads to a slower but more accurate decision; a decrease in \(a\) results in a faster but error-prone decision. The non-decision time \(t\) is the time spent for stimuli encoding or motor execution (\emph{i.e.}, time not used for evidence accumulation). The starting point parameter \(z\) captures a potential initial bias toward one or the other boundary in absence of any stimulus evidence.

To test the interference of disease-related information on the decision process, we built linear models over each RLDDM parameter. We compared models in which we conditioned either none, each or all model's parameters on diagnostic category (group) and image category (neutral, symptom-related). For each model, we computed the Deviance Information Criterion (DIC) and we selected the model with the best trade-off between the fit quality and model complexity (\emph{i.e.}, the model with the lowest DIC).

The following models were examined. The model M1 is a standard RLDDM. The model M2 adds to M1 separate learning rates for positive and negative reinforcements. In the model M3 the \(\alpha^+\) and \(\alpha^-\) parameters are conditioned on diagnostic group. In the model M4 the \(\alpha^+\) and \(\alpha^-\) parameters of M3 are conditioned on both diagnostic group and image category (two neutral images, or one neutral and one symptom-related image). The model M5 adds to M4 the fact that the \(a\) parameter is conditioned on both diagnostic group and image category. The model M6 adds to M5 the fact that the \(v\) parameter is conditioned on both diagnostic group and image category. The model M7 adds to M6 the fact that the \(t\) parameter is conditioned on both diagnostic group and image category. The model M8 adds to M7 the estimate of a possible bias of the \(z\) parameter. All models were estimated with Bayesian methods using weakly informative priors. The winning RLDDM (with lowest DIC) is M7. In the M7 model, the parameters \(\alpha^+\), \(\alpha^-\), \(a\), \(v\), \(t\) (but not \(z\)) conditioned on both diagnostic group and image category.

\begin{longtable}[]{@{}cc@{}}
\toprule()
Model & DIC \\
\midrule()
\endhead
M1 & 103209.264 \\
M2 & 101590.157 \\
M3 & 101613.877 \\
M4 & 99133.675 \\
M5 & 96150.581 \\
M6 & 95434.070 \\
M7 & 92808.856 \\
M8 & 93157.611 \\
\bottomrule()
\end{longtable}

Convergence of Bayesian model parameters was assessed via the Gelman-Rubin statistic. All parameters had \(\hat{R}\) below 1.1 (max = 1.062, mean = 1.002), which does not suggest convergence issues.

To measure the effect of outcome-irrelevant image category on decision-making, we compared, within each diagnostic group, the difference in posterior estimates of the RLDDM parameters between the neutral and symptom-related image conditions. As predicted by hypothesis H1, the decision threshold (\(a\)) was higher for food information compared to neutral information: HC, \(p(a_\text{food} < a_\text{neutral})\) = .0002; AN, \(p(a_\text{food} < a_\text{neutral})\) = .0026; BN, \(p(a_\text{food} < a_\text{neutral})\) = .0140; RI, \(p(a_\text{food} < a_\text{neutral})\) = .0139{]}. Posterior parameters estimates, standard deviation, and 95\% credibility intervals are shown in the following table.

\begin{longtable}[]{@{}lll@{}}
\toprule()
Parameter & Posterior estimate (\(SD\)) & 95\% CI \\
\midrule()
\endhead
a(AN food) & 1.415 (0.039) & 1.339, 1.491 \\
a(AN neutral) & 1.260 (0.038) & 1.186, 1.334 \\
a(BN food) & 1.440 (0.066) & 1.309, 1.567 \\
a(BN neutral) & 1.229 (0.072) & 1.086, 1.368 \\
a(HC food) & 1.340 (0.016) & 1.308, 1.371 \\
a(HC neutral) & 1.258 (0.016) & 1.226, 1.291 \\
a(RI food) & 1.389 (0.039) & 1.312, 1.463 \\
a(RI neutral) & 1.264 (0.042) & 1.183, 1.345 \\
\bottomrule()
\end{longtable}

As predicted by hypothesis H2, our results indicate that, compared to neutral outcome-irrelevant information, decision-making regarding food information decreased the posterior estimate of the learning rate, but only for the AN group when evaluating reward-based learning, \(\alpha^+\) = 0.144 (\(SD\) = 0.092), \(\alpha^+\) = 0.759 (\(SD\) = 0.142), \(p(\alpha^+_\text{food} > \alpha^+_\text{neutral})\) = 0.0013, \(\Delta\) score on a logit scale = 2.939, 95\% CI {[}0.870, 4.975{]}. No other credible differences were found regarding hypothesis H2 (refer to the Supplementary Material for details).

To investigate if the underperformance of AN patients in the RL task was caused by a bias towards non-food choices (regardless of past action-outcome history), we analyzed the frequency of food choices in the PRL block where a food image was paired with a neutral image. As expected based on hypothesis H1, a general bias towards the neutral image was observed: proportion of food choices = 0.484, 95\% CI {[}0.477, 0.492{]}. However, no group-specific bias was found, as indicated by the following three comparisons: AN - HC: prop = -0.00126, 95\% CI {[}-0.0277, 0.0267{]}; BN - HC: prop = 0.01537, 95\% CI {[}-0.0278, 0.0587{]}; BN - AN: prop = 0.01668, 95\% CI {[}-0.0323, 0.0661{]}.

\hypertarget{comorbidity}{%
\subsubsection{Comorbidity}\label{comorbidity}}

Individuals with eating disorders often have comorbid psychiatric conditions, including depression (up to 75\%), bipolar disorder (10\%), anxiety disorders, obsessive-compulsive disorder (40\%), panic disorder (11\%), social anxiety disorder/social phobia, post-traumatic stress disorder (prevalence varies with eating disorder), and substance abuse (15-40\%) -- see Woodside and Staab (2006) for further details. We included patients with comorbidities in our sample to enhance generalizability to the psychiatric population: In the AN group, 16 patients were diagnosed with commorbidity in anxiety disorder, 8 with OCD, 1 in social phobia, and 1 in DAP??; In the BN group, 4 patiens were diagnosed with MOOD?? disorder and 1 with OCD. To study whether the biases present in RL can be attributed to comorbidity, we adapted model M7 to the patients data by distinguishing between patients who have comorbid conditions and those who do not. No credible differences were found in the models' parameters between patients with and without comorbid conditions (refer to the Supplementary Materials for more information).

\hypertarget{discussion}{%
\section{Discussion}\label{discussion}}

\newpage

\hypertarget{references}{%
\section{References}\label{references}}

\hypertarget{refs}{}
\begin{CSLReferences}{1}{0}
\leavevmode\vadjust pre{\hypertarget{ref-bartholdy2016systematic}{}}%
Bartholdy, S., Dalton, B., O'Daly, O. G., Campbell, I. C., \& Schmidt, U. (2016). A systematic review of the relationship between eating, weight and inhibitory control using the stop signal task. \emph{Neuroscience \& Biobehavioral Reviews}, \emph{64}, 35--62.

\leavevmode\vadjust pre{\hypertarget{ref-ben2022working}{}}%
Ben-Artzi, I., Luria, R., \& Shahar, N. (2022). Working memory capacity estimates moderate value learning for outcome-irrelevant features. \emph{Scientific Reports}, \emph{12}(1), 1--10.

\leavevmode\vadjust pre{\hypertarget{ref-bottesi2015}{}}%
Bottesi, G., Ghisi, M., Altoè, G., Conforti, E., Melli, G., \& Sica, C. (2015). {The Italian version of the Depression Anxiety Stress Scales-21: Factor structure and psychometric properties on community and clinical samples}. \emph{Comprehensive Psychiatry}, \emph{60}, 170--181.

\leavevmode\vadjust pre{\hypertarget{ref-caudek2021susceptibility}{}}%
Caudek, C., Sica, C., Cerea, S., Colpizzi, I., \& Stendardi, D. (2021). Susceptibility to eating disorders is associated with cognitive inflexibility in female university students. \emph{Journal of Behavioral and Cognitive Therapy}, \emph{31}(4), 317--328.

\leavevmode\vadjust pre{\hypertarget{ref-caudek2020cognitive}{}}%
Caudek, C., Sica, C., Marchetti, I., Colpizzi, I., \& Stendardi, D. (2020). Cognitive inflexibility specificity for individuals with high levels of obsessive-compulsive symptoms. \emph{Journal of Behavioral and Cognitive Therapy}, \emph{30}(2), 103--113.

\leavevmode\vadjust pre{\hypertarget{ref-chang2021early}{}}%
Chang, P. G., Delgadillo, J., \& Waller, G. (2021). Early response to psychological treatment for eating disorders: A systematic review and meta-analysis. \emph{Clinical Psychology Review}, \emph{86}, 102032.

\leavevmode\vadjust pre{\hypertarget{ref-dottiandlazzari1998}{}}%
Dotti, A., \& Lazzari, R. (1998). Validation and reliability of the italian {EAT-26}. \emph{Eating and Weight Disorders-Studies on Anorexia, Bulimia and Obesity}, \emph{3}(4), 188--194.

\leavevmode\vadjust pre{\hypertarget{ref-DowsonandHenderson2001}{}}%
Dowson, J., \& Henderson, L. (2001). The validity of a short version of the {Body Shape Questionnaire}. \emph{Psychiatry Research}, \emph{102}(3), 263--271.

\leavevmode\vadjust pre{\hypertarget{ref-fleming1984dimensionality}{}}%
Fleming, J. S., \& Courtney, B. E. (1984). The dimensionality of self-esteem: II. Hierarchical facet model for revised measurement scales. \emph{Journal of Personality and Social Psychology}, \emph{46}(2), 404.

\leavevmode\vadjust pre{\hypertarget{ref-frost1990dimensions}{}}%
Frost, R. O., Marten, P., Lahart, C., \& Rosenblate, R. (1990). The dimensions of perfectionism. \emph{Cognitive Therapy and Research}, \emph{14}(5), 449--468.

\leavevmode\vadjust pre{\hypertarget{ref-garner_eating_1982}{}}%
Garner, D. M., Olmsted, M. P., Bohr, Y., \& Garfinkel, P. E. (1982). The {Eating} {Attitudes} {Test}: Psychometric features and clinical correlates. \emph{Psychological Medicine}, \emph{12}(4), 871--878.

\leavevmode\vadjust pre{\hypertarget{ref-guillaume2015impaired}{}}%
Guillaume, S., Gorwood, P., Jollant, F., Van den Eynde, F., Courtet, P., \& Richard-Devantoy, S. (2015). Impaired decision-making in symptomatic anorexia and bulimia nervosa patients: A meta-analysis. \emph{Psychological Medicine}, \emph{45}(16), 3377--3391.

\leavevmode\vadjust pre{\hypertarget{ref-haynos2022beyond}{}}%
Haynos, A. F., Widge, A. S., Anderson, L. M., \& Redish, A. D. (2022). Beyond description and deficits: How computational psychiatry can enhance an understanding of decision-making in anorexia nervosa. \emph{Current Psychiatry Reports}, 1--11.

\leavevmode\vadjust pre{\hypertarget{ref-lombardo2008adattamento}{}}%
Lombardo, C. (2008). Adattamento italiano della multidimensional perfectionism scale (MPS). \emph{Psicoterapia Cognitiva e Comportamentale}, \emph{14}(3), 31--46.

\leavevmode\vadjust pre{\hypertarget{ref-Lovibond1995}{}}%
Lovibond, P. F., \& Lovibond, S. H. (1995). {The structure of negative emotional states: Comparison of the Depression Anxiety Stress Scales (DASS) with the Beck Depression and Anxiety Inventories}. \emph{Behaviour Research and Therapy}, \emph{33}(3), 335--343.

\leavevmode\vadjust pre{\hypertarget{ref-Matera2013}{}}%
Matera, C., Nerini, A., \& Stefanile, C. (2013). The role of peer influence on girls' body dissatisfaction and dieting. \emph{Revue Europ{é}enne De Psychologie Appliqu{é}e/European Review of Applied Psychology}, \emph{63}(2), 67--74.

\leavevmode\vadjust pre{\hypertarget{ref-MattickandClarke1998}{}}%
Mattick, R. P., \& Clarke, J. C. (1998). Development and validation of measures of social phobia scrutiny fear and social interaction anxiety. \emph{Behaviour Research and Therapy}, \emph{36}(4), 455--470.

\leavevmode\vadjust pre{\hypertarget{ref-qian2022update}{}}%
Qian, J., Wu, Y., Liu, F., Zhu, Y., Jin, H., Zhang, H., \ldots{} Yu, D. (2022). An update on the prevalence of eating disorders in the general population: A systematic review and meta-analysis. \emph{Eating and Weight Disorders-Studies on Anorexia, Bulimia and Obesity}, \emph{27}(2), 415--428.

\leavevmode\vadjust pre{\hypertarget{ref-rescorla1972theory}{}}%
Rescorla, R. A. (1972). A theory of pavlovian conditioning: Variations in the effectiveness of reinforcement and nonreinforcement. \emph{Current Research and Theory}, 64--99.

\leavevmode\vadjust pre{\hypertarget{ref-Rosenberg1965}{}}%
Rosenberg, M. (1965). \emph{Society and the adolescent self-image}. Princeton, NJ: Princeton University Press.

\leavevmode\vadjust pre{\hypertarget{ref-schiff2021expectancy}{}}%
Schiff, S., Testa, G., Rusconi, M. L., Angeli, P., \& Mapelli, D. (2021). Expectancy to eat modulates cognitive control and attention toward irrelevant food and non-food images in healthy starving individuals. A behavioral study. \emph{Frontiers in Psychology}, \emph{11}, 3902.

\leavevmode\vadjust pre{\hypertarget{ref-shahar2019credit}{}}%
Shahar, N., Moran, R., Hauser, T. U., Kievit, R. A., McNamee, D., Moutoussis, M., \ldots{} Dolan, R. J. (2019). Credit assignment to state-independent task representations and its relationship with model-based decision making. \emph{Proceedings of the National Academy of Sciences}, \emph{116}(32), 15871--15876.

\leavevmode\vadjust pre{\hypertarget{ref-sicasocialfobia2007}{}}%
Sica, C., Musoni, I., Bisi, B., Lolli, V., \& Sighinolfi, C. (2007). Social phobia scale e social interaction anxiety scale: Traduzione e adattamento italiano. \emph{Bollettino Di Psicologia Applicata}, \emph{252}, 59--71.

\leavevmode\vadjust pre{\hypertarget{ref-udo2019suicide}{}}%
Udo, T., Bitley, S., \& Grilo, C. M. (2019). Suicide attempts in US adults with lifetime DSM-5 eating disorders. \emph{BMC Medicine}, \emph{17}(1), 1--13.

\leavevmode\vadjust pre{\hypertarget{ref-woodside2006management}{}}%
Woodside, B. D., \& Staab, R. (2006). Management of psychiatric comorbidity in anorexia nervosa and bulimia nervosa. \emph{CNS Drugs}, \emph{20}, 655--663.

\leavevmode\vadjust pre{\hypertarget{ref-wu2014set}{}}%
Wu, M., Brockmeyer, T., Hartmann, M., Skunde, M., Herzog, W., \& Friederich, H.-C. (2014). Set-shifting ability across the spectrum of eating disorders and in overweight and obesity: A systematic review and meta-analysis. \emph{Psychological Medicine}, \emph{44}(16), 3365--3385.

\end{CSLReferences}


\end{document}
