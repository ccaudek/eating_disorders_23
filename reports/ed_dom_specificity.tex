% Options for packages loaded elsewhere
\PassOptionsToPackage{unicode}{hyperref}
\PassOptionsToPackage{hyphens}{url}
%
\documentclass[
  man,floatsintext]{apa6}
\usepackage{amsmath,amssymb}
\usepackage{iftex}
\ifPDFTeX
  \usepackage[T1]{fontenc}
  \usepackage[utf8]{inputenc}
  \usepackage{textcomp} % provide euro and other symbols
\else % if luatex or xetex
  \usepackage{unicode-math} % this also loads fontspec
  \defaultfontfeatures{Scale=MatchLowercase}
  \defaultfontfeatures[\rmfamily]{Ligatures=TeX,Scale=1}
\fi
\usepackage{lmodern}
\ifPDFTeX\else
  % xetex/luatex font selection
\fi
% Use upquote if available, for straight quotes in verbatim environments
\IfFileExists{upquote.sty}{\usepackage{upquote}}{}
\IfFileExists{microtype.sty}{% use microtype if available
  \usepackage[]{microtype}
  \UseMicrotypeSet[protrusion]{basicmath} % disable protrusion for tt fonts
}{}
\makeatletter
\@ifundefined{KOMAClassName}{% if non-KOMA class
  \IfFileExists{parskip.sty}{%
    \usepackage{parskip}
  }{% else
    \setlength{\parindent}{0pt}
    \setlength{\parskip}{6pt plus 2pt minus 1pt}}
}{% if KOMA class
  \KOMAoptions{parskip=half}}
\makeatother
\usepackage{xcolor}
\usepackage{longtable,booktabs,array}
\usepackage{calc} % for calculating minipage widths
% Correct order of tables after \paragraph or \subparagraph
\usepackage{etoolbox}
\makeatletter
\patchcmd\longtable{\par}{\if@noskipsec\mbox{}\fi\par}{}{}
\makeatother
% Allow footnotes in longtable head/foot
\IfFileExists{footnotehyper.sty}{\usepackage{footnotehyper}}{\usepackage{footnote}}
\makesavenoteenv{longtable}
\usepackage{graphicx}
\makeatletter
\def\maxwidth{\ifdim\Gin@nat@width>\linewidth\linewidth\else\Gin@nat@width\fi}
\def\maxheight{\ifdim\Gin@nat@height>\textheight\textheight\else\Gin@nat@height\fi}
\makeatother
% Scale images if necessary, so that they will not overflow the page
% margins by default, and it is still possible to overwrite the defaults
% using explicit options in \includegraphics[width, height, ...]{}
\setkeys{Gin}{width=\maxwidth,height=\maxheight,keepaspectratio}
% Set default figure placement to htbp
\makeatletter
\def\fps@figure{htbp}
\makeatother
\setlength{\emergencystretch}{3em} % prevent overfull lines
\providecommand{\tightlist}{%
  \setlength{\itemsep}{0pt}\setlength{\parskip}{0pt}}
\setcounter{secnumdepth}{-\maxdimen} % remove section numbering
% Make \paragraph and \subparagraph free-standing
\ifx\paragraph\undefined\else
  \let\oldparagraph\paragraph
  \renewcommand{\paragraph}[1]{\oldparagraph{#1}\mbox{}}
\fi
\ifx\subparagraph\undefined\else
  \let\oldsubparagraph\subparagraph
  \renewcommand{\subparagraph}[1]{\oldsubparagraph{#1}\mbox{}}
\fi
\newlength{\cslhangindent}
\setlength{\cslhangindent}{1.5em}
\newlength{\csllabelwidth}
\setlength{\csllabelwidth}{3em}
\newlength{\cslentryspacingunit} % times entry-spacing
\setlength{\cslentryspacingunit}{\parskip}
\newenvironment{CSLReferences}[2] % #1 hanging-ident, #2 entry spacing
 {% don't indent paragraphs
  \setlength{\parindent}{0pt}
  % turn on hanging indent if param 1 is 1
  \ifodd #1
  \let\oldpar\par
  \def\par{\hangindent=\cslhangindent\oldpar}
  \fi
  % set entry spacing
  \setlength{\parskip}{#2\cslentryspacingunit}
 }%
 {}
\usepackage{calc}
\newcommand{\CSLBlock}[1]{#1\hfill\break}
\newcommand{\CSLLeftMargin}[1]{\parbox[t]{\csllabelwidth}{#1}}
\newcommand{\CSLRightInline}[1]{\parbox[t]{\linewidth - \csllabelwidth}{#1}\break}
\newcommand{\CSLIndent}[1]{\hspace{\cslhangindent}#1}
\ifLuaTeX
\usepackage[bidi=basic]{babel}
\else
\usepackage[bidi=default]{babel}
\fi
\babelprovide[main,import]{english}
% get rid of language-specific shorthands (see #6817):
\let\LanguageShortHands\languageshorthands
\def\languageshorthands#1{}
% Manuscript styling
\usepackage{upgreek}
\captionsetup{font=singlespacing,justification=justified}

% Table formatting
\usepackage{longtable}
\usepackage{lscape}
% \usepackage[counterclockwise]{rotating}   % Landscape page setup for large tables
\usepackage{multirow}		% Table styling
\usepackage{tabularx}		% Control Column width
\usepackage[flushleft]{threeparttable}	% Allows for three part tables with a specified notes section
\usepackage{threeparttablex}            % Lets threeparttable work with longtable

% Create new environments so endfloat can handle them
% \newenvironment{ltable}
%   {\begin{landscape}\centering\begin{threeparttable}}
%   {\end{threeparttable}\end{landscape}}
\newenvironment{lltable}{\begin{landscape}\centering\begin{ThreePartTable}}{\end{ThreePartTable}\end{landscape}}

% Enables adjusting longtable caption width to table width
% Solution found at http://golatex.de/longtable-mit-caption-so-breit-wie-die-tabelle-t15767.html
\makeatletter
\newcommand\LastLTentrywidth{1em}
\newlength\longtablewidth
\setlength{\longtablewidth}{1in}
\newcommand{\getlongtablewidth}{\begingroup \ifcsname LT@\roman{LT@tables}\endcsname \global\longtablewidth=0pt \renewcommand{\LT@entry}[2]{\global\advance\longtablewidth by ##2\relax\gdef\LastLTentrywidth{##2}}\@nameuse{LT@\roman{LT@tables}} \fi \endgroup}

% \setlength{\parindent}{0.5in}
% \setlength{\parskip}{0pt plus 0pt minus 0pt}

% Overwrite redefinition of paragraph and subparagraph by the default LaTeX template
% See https://github.com/crsh/papaja/issues/292
\makeatletter
\renewcommand{\paragraph}{\@startsection{paragraph}{4}{\parindent}%
  {0\baselineskip \@plus 0.2ex \@minus 0.2ex}%
  {-1em}%
  {\normalfont\normalsize\bfseries\itshape\typesectitle}}

\renewcommand{\subparagraph}[1]{\@startsection{subparagraph}{5}{1em}%
  {0\baselineskip \@plus 0.2ex \@minus 0.2ex}%
  {-\z@\relax}%
  {\normalfont\normalsize\itshape\hspace{\parindent}{#1}\textit{\addperi}}{\relax}}
\makeatother

% \usepackage{etoolbox}
\makeatletter
\patchcmd{\HyOrg@maketitle}
  {\section{\normalfont\normalsize\abstractname}}
  {\section*{\normalfont\normalsize\abstractname}}
  {}{\typeout{Failed to patch abstract.}}
\patchcmd{\HyOrg@maketitle}
  {\section{\protect\normalfont{\@title}}}
  {\section*{\protect\normalfont{\@title}}}
  {}{\typeout{Failed to patch title.}}
\makeatother

\usepackage{xpatch}
\makeatletter
\xapptocmd\appendix
  {\xapptocmd\section
    {\addcontentsline{toc}{section}{\appendixname\ifoneappendix\else~\theappendix\fi\\: #1}}
    {}{\InnerPatchFailed}%
  }
{}{\PatchFailed}
\keywords{keywords\newline\indent Word count: X}
\usepackage{lineno}

\linenumbers
\usepackage{csquotes}
\ifLuaTeX
  \usepackage{selnolig}  % disable illegal ligatures
\fi
\IfFileExists{bookmark.sty}{\usepackage{bookmark}}{\usepackage{hyperref}}
\IfFileExists{xurl.sty}{\usepackage{xurl}}{} % add URL line breaks if available
\urlstyle{same}
\hypersetup{
  pdftitle={Anorexia Nervosa: Symptom-Related Information Alters Decision-Making Policy Despite Outcome Irrelevance},
  pdfauthor={Corrado Caudek1 \& Ernst-August Doelle1,2},
  pdflang={en-EN},
  pdfkeywords={keywords},
  hidelinks,
  pdfcreator={LaTeX via pandoc}}

\title{Anorexia Nervosa: Symptom-Related Information Alters Decision-Making Policy Despite Outcome Irrelevance}
\author{Corrado Caudek\textsuperscript{1} \& Ernst-August Doelle\textsuperscript{1,2}}
\date{}


\shorttitle{Title}

\authornote{

Add complete departmental affiliations for each author here. Each new line herein must be indented, like this line.

Enter author note here.

The authors made the following contributions. Corrado Caudek: Conceptualization, Writing - Original Draft Preparation, Writing - Review \& Editing; Ernst-August Doelle: Writing - Review \& Editing, Supervision.

Correspondence concerning this article should be addressed to Corrado Caudek, Postal address. E-mail: \href{mailto:my@email.com}{\nolinkurl{my@email.com}}

}

\affiliation{\vspace{0.5cm}\textsuperscript{1} Wilhelm-Wundt-University\\\textsuperscript{2} Konstanz Business School}

\abstract{%
One or two sentences providing a \textbf{basic introduction} to the field, comprehensible to a scientist in any discipline.

Two to three sentences of \textbf{more detailed background}, comprehensible to scientists in related disciplines.

One sentence clearly stating the \textbf{general problem} being addressed by this particular study.

One sentence summarizing the main result (with the words ``\textbf{here we show}'' or their equivalent).

Two or three sentences explaining what the \textbf{main result} reveals in direct comparison to what was thought to be the case previously, or how the main result adds to previous knowledge.

One or two sentences to put the results into a more \textbf{general context}.

Two or three sentences to provide a \textbf{broader perspective}, readily comprehensible to a scientist in any discipline.
}



\begin{document}
\maketitle

\hypertarget{introduction}{%
\section{Introduction}\label{introduction}}

Anorexia Nervosa (AN) is one of the most common eating disorders characterized by distorted body perception and pathological weight loss, particularly in its restricting type (R-AN) (American Psychiatric Association, 2022). Lifetime prevalence for AN has been reported at 1.4\% for women and 0.2\% for men (Galmiche, Déchelotte, Lambert, \& Tavolacci, 2019; Smink, Hoeken, \& Hoek, 2013), with a mortality rate that can be as high as 5--20\% (Qian et al., 2022). Treating AN is extremely challenging, highlighting the importance of gaining a deeper understanding of its underlying mechanisms (Chang, Delgadillo, \& Waller, 2021).

Dysfunctional executive processes, including impairments in cognitive inflexibility (Wu et al., 2014), decision-making (Guillaume et al., 2015), and inhibitory control (Bartholdy, Dalton, O'Daly, Campbell, \& Schmidt, 2016), have been proposed as potential risk and maintaining factors for AN. Among these processes, associative learning within the framework of Reinforcement Learning (RL) has received the greatest attention.

It is well-known that AN shows dysfunction reward processing, with reduced subjective reward sensitivity and decreased neural response to rewarding stimuli. Additionally, the processing of aversive stimuli may be also disrupted in AN, leading to elevated harm avoidance, intolerance of uncertainty, anxiety, and oversensitivity to punishment (Fladung, Schulze, Schöll, Bauer, \& Groen, 2013; Jappe et al., 2011; Keating, Tilbrook, Rossell, Enticott, \& Fitzgerald, 2012; O'Hara, Campbell, \& Schmidt, 2015). These factors can contribute to an altered response to negative feedback and a bias towards avoiding aversive outcomes (Jonker, Glashouwer, \& Jong, 2022; Matton, Goossens, Braet, \& Vervaet, 2013). Neuroimaging studies provide evidence of neural dysfunction in AN's response to loss and aversive taste (Bischoff-Grethe et al., 2013; Monteleone et al., 2017; Wagner et al., 2007).

The existing research on reward sensitivity in AN provides valuable insights into how individuals respond to reward and punishment. However, there is a notable lack of evidence regarding potential abnormalities specifically in reinforcement learning processes related to both reward and punishment (Bernardoni et al., 2018; Foerde et al., 2021; Foerde \& Steinglass, 2017). Understanding these processes is crucial because the persistent dietary restriction observed in R-AN, despite negative consequences, may be partially attributed to anomalies in the ability to learn from experience, that is, in reinforcement learning processes (Bischoff-Grethe et al., 2013; Glashouwer, Bloot, Veenstra, Franken, \& Jong, 2014; Harrison, Genders, Davies, Treasure, \& Tchanturia, 2011; Jappe et al., 2011; Matton et al., 2013).

We propose that the inconsistent evidence regarding reinforcement learning abnormalities in AN may be partially attributed to the assumption that reinforcement learning is a unitary process thaat operates uniformly in all conditions. According to such assumption, anomalies in AN concerning RL should be attributed to deficits in this underlying unitary learning process (ref).

However, an alternative perspective suggests that atypical reinforcement learning behavior in AN may be attributed to the influence of contextual factors (Haynos, Widge, Anderson, \& Redish, 2022). According to this hypothesis, contextual factors, including personal characteristics, long-term goals, and situational influences, can have a negative impact on performance in a reinforcement learning task, even if they are not directly related to the task outcome (Caudek, Sica, Cerea, Colpizzi, \& Stendardi, 2021). Individuals with AN may be particularly susceptible to these factors, which can encompass symptom-related information such as food, body weight, and social pressure {[}ref{]}.

This study aims to investigate whether decision-making in AN can be influenced by extraneous contextual factors, even in the absence of any deficit in the underlying RL mechanisms (Haynos et al., 2022). To this purpose, in the present study we utilized a probabilistic associative learning task for comparing RL performance of three groups: individuals with DSM-5 restricting-type AN, healthy controls (HCs), and individuals at risk of developing eating disorders (RIs). The primary focus of our study was to utilize computational models of reinforcement learning to analyze and discern variations in learning outcomes within two distinct contextual conditions: decision-making tasks related to food and those unrelated to food. We posited that the disparities in RL between AN patients and the other two control groups can emerge in the food-related decision-making context, even when no group differences are found in the food-unrelated context. To the best of our knowledge, no prior study has directly compared performance in neutral and food-related conditions for the same patients using a computational approach to RL performance.

\hypertarget{evidence-of-contextual-factors-on-rl-learning-in-an}{%
\subsection{Evidence of contextual factors on RL learning in AN}\label{evidence-of-contextual-factors-on-rl-learning-in-an}}

TODO

The hypothesis of maladaptive associative learning is theoretically appealing, as it offers potential for treatment, but the evidence to support it is inconsistent (for a recent discussion, see Caudek et al., 2021).

The aim of this study is to add to the existing research by investigating if individuals with eating disorders can display impaired decision-making despite having normal cognitive decision-making skills. Specifically, we will ask whether \emph{task-irrelevant} symptom-related information can negatively impact decision-making in EDs, potentially indicating that disordered eating may not stem from deficient decision-making abilities, but rather from external factors like long-term goals, personality traits, etc. affecting their choices. The potential translational impact of this result would be noteworthy, when considering that \ldots{}

RL is the ability to infer causal associations between actions and outcomes in a trial-and-error manner. Learning the consequences of past actions is usually studied in the laboratory with a 2-armed bandit task, where a decision maker is presented with two options. One option has a higher likelihood of winning. The participant must learn which choice will yield the highest reward.

In the 2-armed bandit task, the optimal policy for maximizing long-term rewards is based solely on the history of actions and outcomes. Recent research has shown, however, that human reinforcement learning can be impacted by features unrelated to the outcomes. For instance, a study by Shahar et al. (2019) explored the impact of spatial-motor associations on participant reinforcement learning. Optimal decision making should prioritize the reward regardless of any spatial-motor associations (such as the choice of response key in the previous trial). Instead, Shahar et al. (2019) found that rewards had a greater impact on the probability of selecting one of two images presented in each trial when the chosen image was linked to the same response key in both the \(n\) - 1 and \(n\) trials. This demonstrates that, in the general population, decision making can be influenced by features that have no relation to the outcomes (see also Ben-Artzi, Luria, \& Shahar, 2022).

The evidence that outcome-irrelevant factors impact action value-updating raises the possibility of reevaluating prior reinforcement learning results in EDs. Therefore, subpar decision-making in eating disorders may be linked to these factors rather than being solely considered as a deficit. This external influence could partially account for the inconsistent findings of abnormal decision-making in some ED studies but not in others (for a discussion, see Caudek et al., 2021).

The hypothesis that motivates the present study is that AN patients, due to their strict weight control behavior and emphasis on long-term thinness, and BN patients, due to their impulsivity, will be impacted by interference in their decision-making when faced with a 2-bandit task between a food or non-food item. This suggests that long-term goals in AN or temperamental factors in BN can influence their decision-making process when food is present in the task but outcome-irrelevant, even in the absence of any decision-making deficits (see also Haynos et al., 2022).

Two predictions are made regarding the influence of outcome-irrelevant features on PRL performance. Firstly, we anticipate that both ED patients and healthy controls will process food information with more caution compared to neutral information (the domain-specific cognitive load hypothesis, H1). This outcome, which has not been observed in previous PRL studies, aligns with prior research that has shown differences in attention and cognitive control for food and non-food information in other tasks. (\emph{e.g.}, Schiff, Testa, Rusconi, Angeli, \& Mapelli, 2021). Secondly and more significantly, we expect a decline in the learning rate due to symptom-related interference caused by disease-specific information that is outcome-irrelevant (the domain-specific policy hypothesis, H2).

\hypertarget{implications-for-treatment}{%
\subsection{Implications for treatment}\label{implications-for-treatment}}

The hypothesis of maladaptive RL in AN has potential implications for treatment. For example, Cognitive Remediation Therapy (CRT) has been proposed as an adjunct treatment targeting specific cognitive processes in AN and other eating disorders. CRT involves cognitive exercises and behavioral interventions aimed at increasing central coherence abilities, reducing cognitive and behavioral inflexibility, and enhancing thinking style comprehension (Tchanturia et al., 2010). A key aspect of CRT is to avoid discussing symptom-related themes and instead use neutral stimuli in cognitive and behavioral exercises. This approach aims to develop a therapeutic alliance and to decrease drop-out rates, particularly with AN patients.

However, recent evidence shows that CRT may not consistently improve central coherence abilities, cognitive flexibility, or ED-related symptoms (Hagan et al., 2020; Tchanturia et al., 2017). Trapp et al.~(2022) have proposed improvements to address practical issues encountered in CRT application, questioning the use of neutral stimuli and drawing support from Beck's cognitive theory of depression (Beck et al., 1987). This proposal aligns with the present hypothesis that maladaptive eating behavior in AN may be influenced by contextual factors rather than solely deficits in the RL mechanism. Confirming the hypothesis of contextual learning would therefore have significant treatment implications.

\hypertarget{methods}{%
\section{Methods}\label{methods}}

We report how we determined our sample size, all data exclusions (if any), all manipulations, and all measures in the study.

\hypertarget{participants}{%
\subsection{Participants}\label{participants}}

The final sample consists of 69 female outpatients (acAN N = 40, recAN N = 10, acBN N = 13, recBN = 6) and 222 healthy female controls (HCs). Outpatients met Diagnostic and Statistical Manual of Mental Disorders-5 (DSM-5) (American Psychiatric Association, 2013) criteria for AN or BN. They were recruited from the Specchidacqua Institute, Montecatini (PT), Italy, specialized in Eating Disorders. Eligibility was evaluated by the Mental Health professionals of the Institute, the exclusion criteria were having neurological illness, suicidal ideation, alcohol or drug addiction, or psychosis. The acAN (mean age = 20.5 years, \emph{SD} = 1.13) and acBN (mean age = 23.15 years, \emph{SD} = 1.87) participants were admitted to psychological treatment at Specchidacqua Institute, 45\% of them were also taking antidepressant medication (SSRI), and 38\% reported comorbidity with other psychiatric illnesses (22\% anxiety disorders, 20\% obsessive-compulsive disorder, 9\% mood disorders). Mean Body Mass Index was considerable lower for acAN patients (BMI mean = 18.29kg/m2 ) then acBN (BMI mean = 24.84kg/m2). Recovered outpatients were recruited from the Gruber Residence, Bologna (BO), Italy. To be included in the recovered group, recAN (mean age = 24.1 years, \emph{SD} = 1.8) and recBN (mean age = 29.3 years, \emph{SD} = 2.5) outpatients had to (a) not being seriously underweight (\(BMI \ge 18.5\) kg/m2), (b) not engage in dysfunctional eating behaviors (\emph{e.g.}, restrictive diet or binging/purging) for at least 6 months, and (c) being adherent to the psychological treatment. HC participants were recruited from undergraduate psychology courses at the University of Florence, Italy, or via social networks. To be included in the HC group, participants had to have a normal Body Mass Index (BMI mean = 21.29 kg/m2), have no history of psychiatric illness, and have no diagnosis of Eating Disorders, according to the Eating Attitudes Test-26 {[}EAT-26; Garner, Olmsted, Bohr, and Garfinkel (1982), Dotti and Lazzari (1998){]} score (EAT-26 \textless{} 20). However, 28 out of 222 participants exceeded the EAT-26 cut-off (EAT-26 \textgreater{} 20), meaning the presence of a tendency to eating symptoms. Therefore, the final HC group was composed by 194 participants (mean age = 21.5 years, \emph{SD} = 0.23), and the other 28 were classified as at-risk participants (mean age = 21.28 years, \emph{SD} = 0.55).
All participants were caucasian, right-handed, and were na"ive to the aim of the study.

\hypertarget{material}{%
\subsection{Material}\label{material}}

\hypertarget{clinical-and-demographic-measurements}{%
\subsubsection{Clinical and Demographic Measurements}\label{clinical-and-demographic-measurements}}

The \emph{Eating Attitude Test-26} (EAT-26, Garner et al., 1982) consists of 26 items assessing levels and types of eating disturbances in the past three mouths. The EAT-26 is characterized by three subscales: the Dieting Scale, the Bulimia and Food Preoccupation Scale and the Oral Control Scale. Scores \(\ge 20\) point out the presence of an eating disorder. Respondents are required to rate intensity associated with the items on a 6-point Likert scale (0 = never, rarely, sometimes; 3 = always).
The Italian version of the EAT-26 demonstrated good psychometric properties (Dotti \& Lazzari, 1998). In fact, Cronbach's alpha was high in an undergraduate sample for the Dieting scale (.87), for Bulimia and Food Preoccupation scale (.70), for Oral Control Scale (.62). Cronbach's alpha for the total scores was 0.86.

The \emph{Body Shape Questionnaire-14} {[}BSQ-14; Dowson and Henderson (2001){]} is a 14-item self-report scale assessing the global body satisfaction in the past two weeks. Respondents are required to rate intensity of concerns about own appearance associated with the items on a 6-point Likert scale (1 = never, 6 = always). The Italian version of the BSQ-14 demonstrated good psychometric properties (Matera, Nerini, \& Stefanile, 2013). In the present sample, \(\omega\) = 0.978. For the 40-item BSQ, a score below 80 is considered ``no concern'', a score of 80 to 110 is considered ``slight concern'', a score of 111 to 140 is considered ``moderate concern'', and a score above 140 is considered ``marked concern''.

The \emph{Social Interaction Anxiety Scale} {[}SIAS; Mattick and Clarke (1998){]} is a 20-item self-report questionnaire assessing social interaction anxiety. Respondents are required to rate intensity associated with the items on a 4-point Likert scale from 0 (not at all true) to 4 (extremely true). Higher scores denote greater social interaction anxiety levels. Both original version and the Italian version (Sica, Musoni, Bisi, Lolli, \& Sighinolfi, 2007) show acceptable psychometric properties (in the present sample \(\omega\) = 0.938). Heimberg, Mueller, Holt, Hope, and Liebowitz (1992) have suggested a cut-off of 34 on the 20-item SIAS to denote a clinical level of social anxiety (32.3 for the Italian 19 item version).

The \emph{Depression Anxiety Stress Scale-21} {[}DASS-21; Lovibond and Lovibond (1995){]} is a 21-item self-report measure assessing depression, anxiety, and stress over the previous week. Items are rated on a 4-point scale ranging from 0 (did not apply to me at all) to 3 (applied to me very much). Both the original and the Italian version (Bottesi et al., 2015) demonstrate adequate reliability. In the present sample \(\omega_{\text{anxiety}}\) = 0.875, \(\omega_{\text{depression}}\) = 0.914, \(\omega_{\text{stress}}\) = 0.899; for the total scale, \(\omega\) = 0.945.

The \emph{Rosenberg Self-Esteem Scale} {[}RSES; Rosenberg (1965){]} is a 10-item scale designed to assess person's overall self-esteem. It comprises five straightforwardly worded and five reverse-worded items each rated on a 4-point Likert scale ranging from 4 (strongly agree) to 1 (strongly disagree). Increased values indicate increased self-esteem. In the present sample, \(\omega\) = 0.949.

The \emph{Multidimensional Perfectionism Scale} {[}MPS-F; Frost, Marten, Lahart, and Rosenblate (1990){]} is a 35-item assessing perfectionism tendencies. According to Stöber (1998), MPS-F is composed of four underlying factors: Concerns over Mistakes and Doubts (CMD), Parental Expectations and Criticism (PEC), Personal Standards (PS), and Organization (O). Both the original MPS-F and the Italian version (Lombardo, 2008) demonstrate adequate reliability. In the present sample, \(\omega_{\text{CMD}}\) = 0.919, \(\omega_{\text{PS}}\) = 0.851, \(\omega_{\text{PEPC}}\) = 0.946, \(\omega_{\text{OR}}\) = 0.931; for the total scale, \(\omega\) = 0.932.

Height and weight were measured with a stadiometer and a digital scale, respectively. Estimated IQ was assessed with the Progressive Raven's Matrices Intelligence test.

\hypertarget{procedure}{%
\subsection{Procedure}\label{procedure}}

The study was approved by the Ethical Committee of the University of Florence, and was run in accordance with the Declaration of Helsinki.
Each eligible participant signed the informed consent and agreed to be part of the study.
Both the HCs group and the patients group completed the same tasks.
Data collection started in December, 2020 until June, 2022. We have to deal with COVID-19 restrictions for the most of the time. Thus, we collected data from HCs remotely: we recruited HCs participants by means of social networks or advertisements at the University. Interested people contacted us using the email on the advertisement, then we send them the informed consent, which they had to sign and send it back to us. Individuals that signed the informant consent were tested for eligibility using self-reported measures. Participants who met the inclusion criteria for HCs group, received instructions via email and completed the PRL task remotely. After completing the task, participants had to notify us, so that we can check the correct registration of data.
On the contrary, data collection for the clinical group was in person. We enrolled only eligible patients, selected by the mental health professionals of the Institute. We scheduled two meeting per participants at the Specchidacqua Institute, Montecatini (PT), Italy.\\
On the first session, participants signed the informed consent form and completed a battery of self-report questionnaires. On the second session, participants were asked to complete the PRL task. Data collection required overall 1 hour of their time.

Participants completed a reinforcement learning bandit task in two conditions: neutral (two neutral images on each trial) and symptom-specific (a symptom-specific and a neutral image on each trial). This design allowed us to examine outcome-irrelevant learning associated to a symptom-specific context.

Participants completed a total of 2 blocks of the reinforcement learning task. Each block included a different set of image stimuli and had XX trials. Participants did not received any bonus at the end of the task based on their performance.

For measuring cognitive flexibility, participants completed a computerized Probabilistic Reversal Learning (PRL) task.
There were two blocks of trials including 160 trials each. In one of the two blocks a neutral image (\emph{e.g.}, a lamp) and a symptom-related image (\emph{i.e.}, a piece of cake) were shown together, to test the domain-specificity hypothesis Caudek, Sica, Marchetti, Colpizzi, \& Stendardi, 2020). The other block included neutral images only, as a control task.
In both blocks we asked participants to choose one of two stimuli presented simultaneously on the left and right side of the center of a screen and made their choice with a keypress. They had 3s response time per trial. An image of a euro coin was provided as a reward and a strikethrough image of a euro coin as a punishment. Feedback was presented for 2 s.
The PRL comprises four epochs (\emph{e.g.}, a sequence of trials in which the same image was considered correct) of 40 trials each.
The feedback was probabilistic, which means that for each epoch the correct image was rewarded in the 70\% of the cases, whereas on 30\% of the trials participants received a negative feedback.
As a consequence, the other image provided no-reward 70\% of the time. Both blocks consisted of three rule changes (reversal phase).
Participants' aim was to earn as much money as possible. They were informed that the stimulus-reward contingencies would change, but they were not told how or when it would happen. Total reward earned was shown at the end of each block.
The experiment was controlled by \href{https://www.psytoolkit.org/}{Psytoolkit}.

\hypertarget{data-analysis}{%
\subsection{Data analysis}\label{data-analysis}}

Credible effects were revealed by 95\% credible intervals or by 97.5\% of posterior samples falling above or below 0 when computing proportion of posterior in direction of effect.

\hypertarget{results}{%
\section{Results}\label{results}}

\hypertarget{quality-control}{%
\subsection{Quality Control}\label{quality-control}}

Trials were excluded for extreme RTs (\textless150 ms, \textgreater2500 ms), or if the remaining (log transformed) RT exceeded the participant's mean ± 3S.D. Participants' datasets were excluded if, in any block, there were more than 20 RT outliers, fewer than 24 rich or 7 lean rewards, a rich-to-lean reward ratio lower than 2.5, or lower than 40\% correct accuracy. In Study 1, 258 depressed adults and 36 controls passed the QC criteria. Study 2 data are from participants who passed these QC checks.

\hypertarget{demographic-and-psychopathology-measures}{%
\subsection{Demographic and Psychopathology Measures}\label{demographic-and-psychopathology-measures}}

\hypertarget{estimating-outcome-irrelevant-learning}{%
\subsection{Estimating outcome-irrelevant learning}\label{estimating-outcome-irrelevant-learning}}

\hypertarget{spatial-motor-associations}{%
\subsubsection{Spatial-motor associations}\label{spatial-motor-associations}}

We start by examining the presence of spatial-motor connections in the participants' choices. We successfully replicated the findings of Shahar et al. (2019) and Ben-Artzi et al. (2022). Our results showed robust evidence for spatial-motor outcome-irrelevant learning: the probability of choosing `stay' was higher for `same' (.427) compared to `flipped' (.218) response/key mapping when comparing previously rewarded versus unrewarded responses (posterior \(\beta\) = 0.93, \emph{SE} = 0.06, \(\text{HDI}_{.95}\) = {[}0.81, 1.06{]}; probability of direction (pd) 1.0; 0\% in ROPE (-0.10, 0.10) and Bayes Factor (BF) of \(>\) 100 against the null; Fig. 1). There was no group (HC, AN, BN, RI) \(\times\) previous outcome \(\times\) mapping interaction (see Supplementary Materials).

\hypertarget{reinforcement-learning-and-drift-diffusion-modeling}{%
\subsection{Reinforcement learning and drift diffusion modeling}\label{reinforcement-learning-and-drift-diffusion-modeling}}

To model the two-choice decision (between image A and image B) over time in the PRL task, we used a hierarchical reinforcement learning drift diffusion model (RLDDM), as described in Pedersen, Frank, and Biele (2017) and Pedersen and Frank (2020). The RLDDM was estimated in a hierarchical Bayesian framework using the \(\texttt{HDDMrl}\) module of the \(\texttt{HDDM}\) (version 0.9.7) Python package (Fengler et al., 2021; Wiecki et al., 2013).

By breaking down decision-making task performance into its component processes through cognitive modeling analysis, it becomes possible to identify any deviances in the underlying mechanisms that may not be reflected in the overall task outcome. RLDDM has six basic parameters: positive learning rate (\(alpha^+\)), negative learning rate (\(alpha^-\)), drift rate (\(v\)), decision threshold (\(a\)), non-decision time (\(t\)), and starting point bias (\(z\)) parameters. The \(\alpha\) parameter quantifies the learning rate in the Rescorla-Wagner delta learning rule (Rescorla, 1972); a higher learning rate results in rapid adaptation to reward expectations, while a lower learning rate results in slow adaptation. The parameter \(\alpha^+\) is computed from reinforcements, whereas \(\alpha^+\) is computed from punishments. The drift rate \(v\) is the average speed of evidence accumulation toward one decision. The decision boundary is the distance between two decision thresholds; an increase of \(a\) increases the evidence needed to make a decision. The increase of \(a\) leads to a slower but more accurate decision; a decrease in \(a\) results in a faster but error-prone decision. The non-decision time \(t\) is the time spent for stimuli encoding or motor execution (\emph{i.e.}, time not used for evidence accumulation). The starting point parameter \(z\) captures a potential initial bias toward one or the other boundary in absence of any stimulus evidence.

To test the interference of disease-related information on the decision process, we built linear models over each RLDDM parameter. We compared models in which we conditioned either none, each or all model's parameters on diagnostic category (group) and image category (neutral, symptom-related). For each model, we computed the Deviance Information Criterion (DIC) and we selected the model with the best trade-off between the fit quality and model complexity (\emph{i.e.}, the model with the lowest DIC).

The following models were examined. Model M1 is a standard RLDDM. Model M2 extends M1 by incorporating separate learning rates for positive and negative reinforcements. In Model M3, the \(\alpha^+\) and \(\alpha^-\) parameters are based on the diagnostic group. In Model M4, the \(\alpha^+\) and \(\alpha^-\) parameters of M3 are conditioned on both diagnostic group and image category (two neutral images, or one neutral and one symptom-related image). Model M5 expands upon M4 by considering that the \(a\) parameter may be influenced by both diagnostic group and image category. Model M6 extends M5 by taking into account the possible influence of diagnostic group and image category on the \(v\) parameter. Model M7 builds upon M6 by considering that the \(t\) parameter may depend on both diagnostic group and image category. Finally, Model M8 adds to Model M7 the estimation of a potential bias in the \(z\) parameter. All models were estimated with Bayesian methods using weakly informative priors. The winning RLDDM (with lowest DIC) is M7. In the Model M7, the parameters \(\alpha^+\), \(\alpha^-\), \(a\), \(v\), \(t\) (but not \(z\)) are conditioned on both diagnostic group and image category.

\begin{longtable}[]{@{}cc@{}}
\toprule\noalign{}
Model & DIC \\
\midrule\noalign{}
\endhead
\bottomrule\noalign{}
\endlastfoot
M1 & 103209.264 \\
M2 & 101590.157 \\
M3 & 101613.877 \\
M4 & 99133.675 \\
M5 & 96150.581 \\
M6 & 95434.070 \\
M7 & 92808.856 \\
M8 & 93157.611 \\
\end{longtable}

Convergence of Bayesian model parameters was assessed via the Gelman-Rubin statistic. All parameters had \(\hat{R}\) below 1.1 (max = 1.062, mean = 1.002), which does not suggest convergence issues.

To gauge the impact of outcome-irrelevant image category on decision-making, we contrasted the difference in posterior estimates of the RLDDM parameters between the neutral and symptom-related image conditions within each diagnostic group. As predicted by Hypothesis H1, the decision threshold (\(a\)) was found to be greater for food information than for neutral information: HC, \(p(a_\text{food} < a_\text{neutral})\) = .0002; AN, \(p(a_\text{food} < a_\text{neutral})\) = .0026; BN, \(p(a_\text{food} < a_\text{neutral})\) = .0140; RI, \(p(a_\text{food} < a_\text{neutral})\) = .0139{]}. Posterior parameters estimates, standard deviation, and 95\% credibility intervals are shown in the following table.

\begin{longtable}[]{@{}lll@{}}
\toprule\noalign{}
Parameter & Posterior estimate (\(SD\)) & 95\% CI \\
\midrule\noalign{}
\endhead
\bottomrule\noalign{}
\endlastfoot
a(AN food) & 1.415 (0.039) & 1.339, 1.491 \\
a(AN neutral) & 1.260 (0.038) & 1.186, 1.334 \\
a(BN food) & 1.440 (0.066) & 1.309, 1.567 \\
a(BN neutral) & 1.229 (0.072) & 1.086, 1.368 \\
a(HC food) & 1.340 (0.016) & 1.308, 1.371 \\
a(HC neutral) & 1.258 (0.016) & 1.226, 1.291 \\
a(RI food) & 1.389 (0.039) & 1.312, 1.463 \\
a(RI neutral) & 1.264 (0.042) & 1.183, 1.345 \\
\end{longtable}

As expected by Hypothesis H2, our findings indicate that compared to neutral outcome-irrelevant information, decision-making regarding food information resulted in a lower estimate of the learning rate, but only for the AN group when evaluating reward-based learning, \(\alpha^+\) = 0.144 (\(SD\) = 0.092), \(\alpha^+\) = 0.759 (\(SD\) = 0.142), \(p(\alpha^+_\text{food} > \alpha^+_\text{neutral})\) = 0.0013, \(\Delta\) score on a logit scale = 2.939, 95\% CI {[}0.870, 4.975{]}. No other credible differences were found regarding Hypothesis H2 (see the Supplementary Material for details).

\hypertarget{biased-choices}{%
\subsection{Biased choices}\label{biased-choices}}

To determine if the subpar performance of AN patients in the RL task was due to a bias towards non-food choices (independent of past action-outcome history), we examined the frequency of food choices in the PRL blocks where a food image was paired with a neutral image. As anticipated based on Hypothesis H1, a bias against the food image was observed: proportion of food choices = 0.484, 95\% CI {[}0.477, 0.492{]}. However, no group-specific bias was detected, as evidenced by the following three comparisons: AN - HC: prop = -0.002, 95\% CI {[}-0.029, 0.026{]}; BN - HC: prop = 0.015, 95\% CI {[}-0.029, 0.056{]}; BN - AN: prop = 0.017, 95\% CI {[}-0.035, 0.064{]}; RI- HC: prop = -0.007, 95\% CI {[}-0.031, 0.016{]}.

\hypertarget{comorbidity}{%
\subsection{Comorbidity}\label{comorbidity}}

Individuals with eating disorders often have comorbid psychiatric conditions, including depression (up to 75\%), bipolar disorder (10\%), anxiety disorders, obsessive-compulsive disorder (40\%), panic disorder (11\%), social anxiety disorder/social phobia, post-traumatic stress disorder (prevalence varies with eating disorder), and substance abuse (15-40\%) -- see Woodside and Staab (2006) for further details. In this study, we included patients with comorbidities in our sample in order to increase the generalizability of our findings to the broader psychiatric population: 16 patients in the AN group were diagnosed with comorbid anxiety disorder, 8 with OCD, 1 with social phobia, and 1 with DAP; in the BN group, 4 patients were diagnosed with mood disorder and 1 with OCD. Comorbid diagnoses were determined using the DSM-V criteria during psychiatric evaluations spanning a minimum of one year, while the absence of comorbidities was evaluated using the same methods within a comparable timeframe. To determine if the lower learning rate observed in the AN group could be due to comorbidity, we utilized model M7 on the patient data by separating patients into groups with and without comorbid conditions. No credible differences were identified in the parameters of the models between patients with and without comorbid conditions (see Supplementary Materials for additional information).

\hypertarget{discussion}{%
\section{Discussion}\label{discussion}}

There is a growing consensus that the reward and punishment processes in AN are not a generic process, but instead are influenced by complex interactions between various stimulus properties (such as the type of reward/punishment cue) and contextual factors {[}such as long-term objectives, personality traits, temperamental dispositions, and physiological states like hunger, etc.). A recent comprehensive review by Haynos, Lavender, Nelson, Crow, and Peterson (2020) showed that the manner in which AN patients perceive their experiences as rewarding or punishing is influenced by factors such as the degree of predictability, controllability, immediacy, and effort. For example, behaviors associated with AN that are predictable, controllable, and immediate (such as calorie counting or purging) may become rewarding to the individual, providing a sense of control and accomplishment. On the other hand, behaviors that are unpredictable and uncontrollable (such as social outcomes) may be perceived as punishing, increasing anxiety and distress.

Most of these previous studies have mainly explored the subjective value assigned to various experiences by AN patients, which can be perceived as either rewarding or punishing, despite not inherently having these properties. In contrast, the current study examine the effect of contextual factors on the learning mechanism that blends past experiences of clearly defined reward and punishment.

The purpose of this study was to examine the impact of symptom-related information (irrelevant to the task outcome) on the performance of AN and BN patients in an associative learning task. Previous research has shown that outcome-irrelevant information can negatively impact reward learning in the general population. Here, we replicated the findings of Shahar et al. (2019) that image/effector response mapping influences associative learning in a PRL task when only image identity predicts the reward, in all our groups of HCs, AN patients, BN patients, and RI patients. More notably, we discovered that AN patients had a slower learning rate from rewards when image identity provided food information. This was shown by a decrease in the \(\alpha^+\) parameter (which measures the rate of learning from positive feedback) of the RLDDM model, compared to HCs (Pedersen \& Frank, 2020). Instead, when image identity was unrelated to food, there was no difference in the rate of value update between AN patients and HCs.

We also found that AN patients demonstrated a slower rate of learning from positive feedback when food information was provided through image identity, compared to BN patients. Conversely, no significant differences were observed when the image identity was unrelated to food. These findings replicate previous reports that AN and BN patients exhibit divergent anomalies in decision making (\emph{e.g.}, Chan et al., 2014), but also emphasize that these variations are more pronounced when considering the processing of information related to the condition.

The present results are relevant for the current debate on the role of maladaptive reward and punishment processing in AN. Current theories propose that AN is characterized by a combination of reduced sensitivity to reward and increased sensitivity to punishment, leading to an imbalance in reward processing. This imbalance is thought to result in decreased interest in food rewards and increased control over food intake, contributing to the persistence of AN symptoms. Additionally, heightened punishment sensitivity may contribute to AN by promoting avoidance of food and weight gain, which may be perceived as aversive. However, as Haynos et al. (2020) points out, such characterization of AN as having distorted reward and punishment processing, which is a domain-general description, is inadequate because it does not consider the differences in response depending on the particular characteristics of the cues involved. In their literature review, Haynos et al. (2020) show that current evidence does not indicate a universal shortfall in AN reward and punishment processing. Rather, there seem to be an inappropriate interpretation of what constitutes a reward or punishment in various contexts and for different stimuli and decisions. Behaviors that initially may not be considered rewards or punishments can eventually become associated with either positive or negative reactions, leading them to serve as a form of reward or punishment.

For instance, Haynos et al. (2020) posits that restrictive eating cues, a precursor of AN, can be linked to reward responses in AN. This hypothesis is supported by ecological momentary assessment (EMA) studies that examine affective patterns in relation to disordered eating. These studies have shown higher positive affect and lower negative affect before, during, and after restrictive eating episodes in AN compared to normal meals (Fitzsimmons-Craft et al., 2015) and subsequent reductions in guilt in AN and increased self-assurance for individuals with AN-R (Haynos et al., 2017).
These findings indicate that restrictive eating is linked to desirable emotional outcomes in AN and, thus, can be understood as rewarding. Although decreased sensitivity to reward in AN has been documented in some contexts, such as individuals with AN scoring lower on sensation-seeking measures that gauge reactions to immediate novel rewards compared to healthy individuals and those with bulimia nervosa (BN) or binge eating disorder (BED; Matton, Goossens, Vervaet, \& Braet, 2015; Rotella et al., 2018), this does not indicate that a reduced sensitivity to reward is evident across all contexts. For instance, the rewarding nature of restrictive eating is not reflected in this reduced sensitivity. The review by Haynos et al. (2020) offers several additional examples of cues, contexts, or decisions that may only be associated with reward or punishment if they are viewed in the context of the ultimate objectives of AN (i.e., thinness). This way of thinking is very much in line with the present results. What the present study adds to this previous theoretical proposal is that previous evidence of domain-specificity of reward and punishment processing in AN have only been provided in an indirect form, that is, in terms of the re-interpretation of cues and consequences of actions in the context of an overarching long-term goal; instead, the present study, for the first time, addresses this issue in a direct manner within the context of associative learning in which reward and punishment are direct consequences of choices.

Other recent studies have examined the issue of the domain-specificity of maladaptive associative learning in eating disorders. One task that has been specifically devised for this purpose is the two-step Markov decision task, which differentiates between automatic or habitual (model-free) and controlled or goal-directed (model-based) learning. For example, Foerde et al. (2021) and Onysk and Seriès (2022) both conducted similar experiments using this task, with Foerde et al. (2021) comparing a monetary two-step task and a food two-step task, and Onysk and Seriès (2022) using stimuli unrelated to food or body images (pirate ships and treasure chests) with rewards associated with body image dissatisfaction. The results of these experiments showed that individuals with AN displayed a stronger preference for habitual control over goal-directed control across domains compared to healthy controls, but there were no differences in the learning rate. However, the primary aim of the two-step experiments was to determine whether the participants' decision-making strategy was influenced by the context or solely based on the previous feedback received, regardless of the context. The results showed that AN patients had difficulty adapting to changing contexts compared to healthy controls (HCs). Furthermore, the experiments did not reveal any differences in the impact of the context (food-related or neutral) on decision making in AN. More importantly, the two-step task did not uncover any difference in the learning rate of AN patients compared to healthy controls (HCs), as a function of the context. In contrast, our results indicate that the learning process itself, particularly the rate at which values are updated, is influenced by information related to the disease, even when such information is not relevant to the outcome.

From a translational perspective, our findings suggest that, at the stage of the disease currently examined, AN patients exhibit maladaptive learning only in certain contexts, and this appears to be influenced by extraneous variables. This is particularly evident in the current study, where the experimental variable (the image identity in the PRL task) has no bearing on the outcome. These results imply that clinical interventions at the present stage of the disease should not concentrate on fixing a seemingly faulty associative learning mechanism. Instead, attention should be directed towards reducing the influence of disruptive factors that hinder the performance of intact associative learning capabilities.

There remain questions for future research. (1) For example, we used images of a one euro coin or a barred representation of a one euro coin to symbolize rewards and punishments, respectively. But such rewards and punishments are only symbolic and the question remains as to what happens when the rewards and punishments are concrete and not symbolic. Yet, these rewards and punishments were merely symbolic, and the question remains as to what happens when the rewards and punishments are actual and not symbolic. Moreover, the subjective value of one euro, or the loss of one euro, is not constant for all participants. Furthermore, the subjective worth of one euro or the loss of one euro is not uniform across all participants. Determining the equivalence of subjective values for rewards and punishments could be a worthwhile objective for future studies. (2) Our study only included AN patients who were not in the most severe stage of the illness, as they were recruited from a center for individuals seeking voluntary medical and psychological support. We did not consider AN patients who are hospitalized due to the life-threatening nature of their illness. It is possible that at the later stage of the illness, the associative learning abilities, which were shown to be preserved in the present sample under neutral conditions, may become impaired. (3) We observed no difference in the choice behavior of AN patients (as measured by relative frequency of image choices) when they were asked to select between a neutral image and a food image. However, when compared to the situation where they had to choose between two neutral images, this condition did result in a slower learning rate and lower decision threshold for AN patients, as compared to healthy controls, according to the RLDDM model. It is possible that the higher ``salience'' of food images compared to neutral images may be better captured by other measures, such as fixation length or number of fixations, rather than just by the relative frequency of image choices. This could be a topic for future exploration. (4) In our study, we excluded women under the age of 18. However, this age range is a critical period, as the onset of AN during this stage may have a more profound impact on associative learning, given that cognitive development is ongoing and protective factors are less developed. Future studies should take this into consideration.

\newpage

\hypertarget{references}{%
\section{References}\label{references}}

\hypertarget{refs}{}
\begin{CSLReferences}{1}{0}
\leavevmode\vadjust pre{\hypertarget{ref-dsm5tr}{}}%
American Psychiatric Association. (2022). \emph{{Diagnostic and Statistical Manual of Mental Disorders}} (5th ed., Text Revision). Arlington, VA: {American Psychiatric Publishing}.

\leavevmode\vadjust pre{\hypertarget{ref-bartholdy2016systematic}{}}%
Bartholdy, S., Dalton, B., O'Daly, O. G., Campbell, I. C., \& Schmidt, U. (2016). A systematic review of the relationship between eating, weight and inhibitory control using the stop signal task. \emph{Neuroscience \& Biobehavioral Reviews}, \emph{64}, 35--62.

\leavevmode\vadjust pre{\hypertarget{ref-ben2022working}{}}%
Ben-Artzi, I., Luria, R., \& Shahar, N. (2022). Working memory capacity estimates moderate value learning for outcome-irrelevant features. \emph{Scientific Reports}, \emph{12}(1), 1--10.

\leavevmode\vadjust pre{\hypertarget{ref-bernardoni_nutritional_2018}{}}%
Bernardoni, F., King, J. A., Geisler, D., Birkenstock, J., Tam, F. I., Weidner, K., \ldots{} Ehrlich, S. (2018). Nutritional status affects cortical folding: {Lessons} learned from {Anorexia} {Nervosa}. \emph{Biological Psychiatry}, \emph{84}(9), 692--701.

\leavevmode\vadjust pre{\hypertarget{ref-bischoff2013altered}{}}%
Bischoff-Grethe, A., McCurdy, D., Grenesko-Stevens, E., Irvine, L. E. Z., Wagner, A., Yau, W.-Y. W., et al.others. (2013). Altered brain response to reward and punishment in adolescents with anorexia nervosa. \emph{Psychiatry Research: Neuroimaging}, \emph{214}(3), 331--340.

\leavevmode\vadjust pre{\hypertarget{ref-bottesi2015}{}}%
Bottesi, G., Ghisi, M., Altoè, G., Conforti, E., Melli, G., \& Sica, C. (2015). {The Italian version of the Depression Anxiety Stress Scales-21: Factor structure and psychometric properties on community and clinical samples}. \emph{Comprehensive Psychiatry}, \emph{60}, 170--181.

\leavevmode\vadjust pre{\hypertarget{ref-caudek2021susceptibility}{}}%
Caudek, C., Sica, C., Cerea, S., Colpizzi, I., \& Stendardi, D. (2021). Susceptibility to eating disorders is associated with cognitive inflexibility in female university students. \emph{Journal of Behavioral and Cognitive Therapy}, \emph{31}(4), 317--328.

\leavevmode\vadjust pre{\hypertarget{ref-caudek2020cognitive}{}}%
Caudek, C., Sica, C., Marchetti, I., Colpizzi, I., \& Stendardi, D. (2020). Cognitive inflexibility specificity for individuals with high levels of obsessive-compulsive symptoms. \emph{Journal of Behavioral and Cognitive Therapy}, \emph{30}(2), 103--113.

\leavevmode\vadjust pre{\hypertarget{ref-chan2014differential}{}}%
Chan, T. W. S., Ahn, W.-Y., Bates, J. E., Busemeyer, J. R., Guillaume, S., Redgrave, G. W., \ldots{} Courtet, P. (2014). Differential impairments underlying decision making in anorexia nervosa and bulimia nervosa: A cognitive modeling analysis. \emph{International Journal of Eating Disorders}, \emph{47}(2), 157--167.

\leavevmode\vadjust pre{\hypertarget{ref-chang2021early}{}}%
Chang, P. G., Delgadillo, J., \& Waller, G. (2021). Early response to psychological treatment for eating disorders: A systematic review and meta-analysis. \emph{Clinical Psychology Review}, \emph{86}, 102032.

\leavevmode\vadjust pre{\hypertarget{ref-dottiandlazzari1998}{}}%
Dotti, A., \& Lazzari, R. (1998). Validation and reliability of the italian {EAT-26}. \emph{Eating and Weight Disorders-Studies on Anorexia, Bulimia and Obesity}, \emph{3}(4), 188--194.

\leavevmode\vadjust pre{\hypertarget{ref-DowsonandHenderson2001}{}}%
Dowson, J., \& Henderson, L. (2001). The validity of a short version of the {Body Shape Questionnaire}. \emph{Psychiatry Research}, \emph{102}(3), 263--271.

\leavevmode\vadjust pre{\hypertarget{ref-fladung2013role}{}}%
Fladung, A.-K., Schulze, U. M., Schöll, F., Bauer, K., \& Groen, G. (2013). Role of the ventral striatum in developing anorexia nervosa. \emph{Translational Psychiatry}, \emph{3}(10), e315--e315.

\leavevmode\vadjust pre{\hypertarget{ref-foerde2021deficient}{}}%
Foerde, K., Daw, N. D., Rufin, T., Walsh, B. T., Shohamy, D., \& Steinglass, J. E. (2021). Deficient goal-directed control in a population characterized by extreme goal pursuit. \emph{Journal of Cognitive Neuroscience}, \emph{33}(3), 463--481.

\leavevmode\vadjust pre{\hypertarget{ref-foerde2017d}{}}%
Foerde, K., \& Steinglass, J. E. (2017). Decreased feedback learning in anorexia nervosa persists after weight restoration. \emph{International Journal of Eating Disorders}, \emph{50}(4), 415--423.

\leavevmode\vadjust pre{\hypertarget{ref-frost1990dimensions}{}}%
Frost, R. O., Marten, P., Lahart, C., \& Rosenblate, R. (1990). The dimensions of perfectionism. \emph{Cognitive Therapy and Research}, \emph{14}(5), 449--468.

\leavevmode\vadjust pre{\hypertarget{ref-galmiche2019prevalence}{}}%
Galmiche, M., Déchelotte, P., Lambert, G., \& Tavolacci, M. P. (2019). Prevalence of eating disorders over the 2000--2018 period: A systematic literature review. \emph{The American Journal of Clinical Nutrition}, \emph{109}(5), 1402--1413.

\leavevmode\vadjust pre{\hypertarget{ref-garner_eating_1982}{}}%
Garner, D. M., Olmsted, M. P., Bohr, Y., \& Garfinkel, P. E. (1982). The {Eating} {Attitudes} {Test}: Psychometric features and clinical correlates. \emph{Psychological Medicine}, \emph{12}(4), 871--878.

\leavevmode\vadjust pre{\hypertarget{ref-glashouwer2014heightened}{}}%
Glashouwer, K. A., Bloot, L., Veenstra, E. M., Franken, I. H., \& Jong, P. J. de. (2014). Heightened sensitivity to punishment and reward in anorexia nervosa. \emph{Appetite}, \emph{75}, 97--102.

\leavevmode\vadjust pre{\hypertarget{ref-guillaume2015impaired}{}}%
Guillaume, S., Gorwood, P., Jollant, F., Van den Eynde, F., Courtet, P., \& Richard-Devantoy, S. (2015). Impaired decision-making in symptomatic anorexia and bulimia nervosa patients: A meta-analysis. \emph{Psychological Medicine}, \emph{45}(16), 3377--3391.

\leavevmode\vadjust pre{\hypertarget{ref-harrison2011experimental}{}}%
Harrison, A., Genders, R., Davies, H., Treasure, J., \& Tchanturia, K. (2011). Experimental measurement of the regulation of anger and aggression in women with anorexia nervosa. \emph{Clinical Psychology \& Psychotherapy}, \emph{18}(6), 445--452.

\leavevmode\vadjust pre{\hypertarget{ref-haynos2020moving}{}}%
Haynos, A. F., Lavender, J. M., Nelson, J., Crow, S. J., \& Peterson, C. B. (2020). Moving towards specificity: A systematic review of cue features associated with reward and punishment in anorexia nervosa. \emph{Clinical Psychology Review}, \emph{79}, 101872.

\leavevmode\vadjust pre{\hypertarget{ref-haynos2022beyond}{}}%
Haynos, A. F., Widge, A. S., Anderson, L. M., \& Redish, A. D. (2022). Beyond description and deficits: How computational psychiatry can enhance an understanding of decision-making in anorexia nervosa. \emph{Current Psychiatry Reports}, 1--11.

\leavevmode\vadjust pre{\hypertarget{ref-heimberg1992assessment}{}}%
Heimberg, R. G., Mueller, G. P., Holt, C. S., Hope, D. A., \& Liebowitz, M. R. (1992). Assessment of anxiety in social interaction and being observed by others: The social interaction anxiety scale and the social phobia scale. \emph{Behavior Therapy}, \emph{23}(1), 53--73.

\leavevmode\vadjust pre{\hypertarget{ref-jappe2011heightened}{}}%
Jappe, L. M., Frank, G. K., Shott, M. E., Rollin, M. D., Pryor, T., Hagman, J. O., \ldots{} Davis, E. (2011). Heightened sensitivity to reward and punishment in anorexia nervosa. \emph{International Journal of Eating Disorders}, \emph{44}(4), 317--324.

\leavevmode\vadjust pre{\hypertarget{ref-jonker2022punishment}{}}%
Jonker, N. C., Glashouwer, K. A., \& Jong, P. J. de. (2022). Punishment sensitivity and the persistence of anorexia nervosa: High punishment sensitivity is related to a less favorable course of anorexia nervosa. \emph{International Journal of Eating Disorders}, \emph{55}(5), 697--702.

\leavevmode\vadjust pre{\hypertarget{ref-keating2012reward}{}}%
Keating, C., Tilbrook, A. J., Rossell, S. L., Enticott, P. G., \& Fitzgerald, P. B. (2012). Reward processing in anorexia nervosa. \emph{Neuropsychologia}, \emph{50}(5), 567--575.

\leavevmode\vadjust pre{\hypertarget{ref-lombardo2008adattamento}{}}%
Lombardo, C. (2008). Adattamento italiano della multidimensional perfectionism scale (MPS). \emph{Psicoterapia Cognitiva e Comportamentale}, \emph{14}(3), 31--46.

\leavevmode\vadjust pre{\hypertarget{ref-Lovibond1995}{}}%
Lovibond, P. F., \& Lovibond, S. H. (1995). {The structure of negative emotional states: Comparison of the Depression Anxiety Stress Scales (DASS) with the Beck Depression and Anxiety Inventories}. \emph{Behaviour Research and Therapy}, \emph{33}(3), 335--343.

\leavevmode\vadjust pre{\hypertarget{ref-Matera2013}{}}%
Matera, C., Nerini, A., \& Stefanile, C. (2013). The role of peer influence on girls' body dissatisfaction and dieting. \emph{Revue Europ{é}enne De Psychologie Appliqu{é}e/European Review of Applied Psychology}, \emph{63}(2), 67--74.

\leavevmode\vadjust pre{\hypertarget{ref-MattickandClarke1998}{}}%
Mattick, R. P., \& Clarke, J. C. (1998). Development and validation of measures of social phobia scrutiny fear and social interaction anxiety. \emph{Behaviour Research and Therapy}, \emph{36}(4), 455--470.

\leavevmode\vadjust pre{\hypertarget{ref-matton2013punishment}{}}%
Matton, A., Goossens, L., Braet, C., \& Vervaet, M. (2013). Punishment and reward sensitivity: Are naturally occurring clusters in these traits related to eating and weight problems in adolescents? \emph{European Eating Disorders Review}, \emph{21}(3), 184--194.

\leavevmode\vadjust pre{\hypertarget{ref-monteleone2017altered}{}}%
Monteleone, A. M., Monteleone, P., Esposito, F., Prinster, A., Volpe, U., Cantone, E., et al.others. (2017). Altered processing of rewarding and aversive basic taste stimuli in symptomatic women with anorexia nervosa and bulimia nervosa: An fMRI study. \emph{Journal of Psychiatric Research}, \emph{90}, 94--101.

\leavevmode\vadjust pre{\hypertarget{ref-o2015reward}{}}%
O'Hara, C. B., Campbell, I. C., \& Schmidt, U. (2015). A reward-centred model of anorexia nervosa: A focussed narrative review of the neurological and psychophysiological literature. \emph{Neuroscience \& Biobehavioral Reviews}, \emph{52}, 131--152.

\leavevmode\vadjust pre{\hypertarget{ref-onysk2022effect}{}}%
Onysk, J., \& Seriès, P. (2022). The effect of body image dissatisfaction on goal-directed decision making in a population marked by negative appearance beliefs and disordered eating. \emph{Plos One}, \emph{17}(11), e0276750.

\leavevmode\vadjust pre{\hypertarget{ref-pedersen2020simultaneous}{}}%
Pedersen, M. L., \& Frank, M. J. (2020). Simultaneous hierarchical bayesian parameter estimation for reinforcement learning and drift diffusion models: A tutorial and links to neural data. \emph{Computational Brain \& Behavior}, \emph{3}, 458--471.

\leavevmode\vadjust pre{\hypertarget{ref-pedersen2017drift}{}}%
Pedersen, M. L., Frank, M. J., \& Biele, G. (2017). The drift diffusion model as the choice rule in reinforcement learning. \emph{Psychonomic Bulletin \& Review}, \emph{24}, 1234--1251.

\leavevmode\vadjust pre{\hypertarget{ref-qian2022update}{}}%
Qian, J., Wu, Y., Liu, F., Zhu, Y., Jin, H., Zhang, H., \ldots{} Yu, D. (2022). An update on the prevalence of eating disorders in the general population: A systematic review and meta-analysis. \emph{Eating and Weight Disorders-Studies on Anorexia, Bulimia and Obesity}, \emph{27}(2), 415--428.

\leavevmode\vadjust pre{\hypertarget{ref-rescorla1972theory}{}}%
Rescorla, R. A. (1972). A theory of pavlovian conditioning: Variations in the effectiveness of reinforcement and nonreinforcement. \emph{Current Research and Theory}, 64--99.

\leavevmode\vadjust pre{\hypertarget{ref-Rosenberg1965}{}}%
Rosenberg, M. (1965). \emph{Society and the adolescent self-image}. Princeton, NJ: Princeton University Press.

\leavevmode\vadjust pre{\hypertarget{ref-schiff2021expectancy}{}}%
Schiff, S., Testa, G., Rusconi, M. L., Angeli, P., \& Mapelli, D. (2021). Expectancy to eat modulates cognitive control and attention toward irrelevant food and non-food images in healthy starving individuals. A behavioral study. \emph{Frontiers in Psychology}, \emph{11}, 3902.

\leavevmode\vadjust pre{\hypertarget{ref-shahar2019credit}{}}%
Shahar, N., Moran, R., Hauser, T. U., Kievit, R. A., McNamee, D., Moutoussis, M., \ldots{} Dolan, R. J. (2019). Credit assignment to state-independent task representations and its relationship with model-based decision making. \emph{Proceedings of the National Academy of Sciences}, \emph{116}(32), 15871--15876.

\leavevmode\vadjust pre{\hypertarget{ref-sicasocialfobia2007}{}}%
Sica, C., Musoni, I., Bisi, B., Lolli, V., \& Sighinolfi, C. (2007). Social phobia scale e social interaction anxiety scale: Traduzione e adattamento italiano. \emph{Bollettino Di Psicologia Applicata}, \emph{252}, 59--71.

\leavevmode\vadjust pre{\hypertarget{ref-smink2013epidemiology}{}}%
Smink, F. R., Hoeken, D. van, \& Hoek, H. W. (2013). Epidemiology, course, and outcome of eating disorders. \emph{Current Opinion in Psychiatry}, \emph{26}(6), 543--548.

\leavevmode\vadjust pre{\hypertarget{ref-stober1998frost}{}}%
Stöber, J. (1998). The frost multidimensional perfectionism scale revisited: More perfect with four (instead of six) dimensions. \emph{Personality and Individual Differences}, \emph{24}(4), 481--491.

\leavevmode\vadjust pre{\hypertarget{ref-wagner2007altered}{}}%
Wagner, A., Aizenstein, H., Venkatraman, V. K., Fudge, J., May, J. C., Mazurkewicz, L., et al.others. (2007). Altered reward processing in women recovered from anorexia nervosa. \emph{American Journal of Psychiatry}, \emph{164}(12), 1842--1849.

\leavevmode\vadjust pre{\hypertarget{ref-woodside2006management}{}}%
Woodside, B. D., \& Staab, R. (2006). Management of psychiatric comorbidity in anorexia nervosa and bulimia nervosa. \emph{CNS Drugs}, \emph{20}, 655--663.

\leavevmode\vadjust pre{\hypertarget{ref-wu2014set}{}}%
Wu, M., Brockmeyer, T., Hartmann, M., Skunde, M., Herzog, W., \& Friederich, H.-C. (2014). Set-shifting ability across the spectrum of eating disorders and in overweight and obesity: A systematic review and meta-analysis. \emph{Psychological Medicine}, \emph{44}(16), 3365--3385.

\end{CSLReferences}


\end{document}
